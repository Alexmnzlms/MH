\chapter{Descripción de los algoritmos de comparación}
%Descripción en pseudocódigo de los algoritmos de comparación.
\section{K-medias Restringido Débil}
\subsection{Función objetivo}
El objetivo del algoritmo K-medias Restringido Débil --- de heurística greedy --- Es minimizar tanto la desviación general del problema como el numero de restricciones que no se satisfacen (infactibilidad). \\
\\
$ F_{objetivo} = \bar{C} + infeasibility $

\subsection{Descripción del algoritmo}
El funcionamiento de este algoritmo funciona de la siguiente manera:\\
Se presupone que los valores de los centroides antes de comenzar con la ejecución son aleatorios.\\
Se establece un orden aleatorio para recorrer los nodos, pero que se mantenga en todas las iteraciones del algoritmo. Por cada iteración, se asigna a cada elemento $ x_{i} \in X $ un cluster $ c_{i} $. El criterio de asignación es siempre el mismo: se asigna al cluster que menos infactibilidad provoque y en caso de todas las asignaciones provoquen la misma infactibilidad, se asignará al cluster cuyo centroide  $ \mu_{i} $ sea más cercano.
El algoritmo iterara hasta que no se produzca un cambio en el estado de la solución --- en la matriz de clusters --- y entonces terminará.\\
Es importante remarcar que es posible que el algoritmo se quede iterando de manera infinita si entra en un ciclo de asignaciones, esto se puede evitar eligiendo la semilla de generación de números aleatorios de manera correcta.

\subsection{Pseudocódigo del algoritmo}
\begin{minted}
[fontsize=\footnotesize, linenos]
{cpp}
int: i = 0
bool: cambio_c
int: vector[] rsi
double: matriz[][] solucion_ant

Guardar en solucion_ant la matriz cluster

Para cada numero de 0...i:
   Añadir numero a vector rsi
Ordenar rsi aleatoriamente

Hacer:
   cambio_c = falso
   Para cada indice j en rsi:
      asignar_cluster(j)
   Para cada cluster c:
      Si fila c de solucion_ant es distinta a la fila c de clusters:
         calcular_centroide(c)
         cambio_c = verdadero
   Guardar en solucion_ant la matriz cluster

   Si cambio_c:
      Vaciar los clusters

   i++
Mientras: cambio_c

desviacion_general()
f_objetivo = desv_gen + infactibilidad
\end{minted}

\subsection{Operadores propios}
\begin{itemize}
   \item Operador \textbf{asignar\_cluster(elemento n)}: Este operador asigna el elemento n a un cluster siguiendo el criterio de asignación: De los que menos infactibilidad provoquen, el que tenga la menor distancia.
   \begin{minted}
   [fontsize=\footnotesize, linenos]
   {cpp}
   int: c, r_min, d_min, d = 0
   Pareja(int,int): vector[] r

   Para cada cluster i:
      Añadir la pareja (restricciones_incumplidas(n,i) , i) a r

   Ordenar r en orden ascendente

   r_min = r[0]
   d_min = Infinito
   Para cada indice j de 0...tamaño(r)
      d = distancia_nodo_cluster(n, r[j].segundo)
      Si d < d_min:
         d_min = d
         c = r[j].segundo

   infactibilidad += r_min
   Añadir n al cluster c
   \end{minted}
\end{itemize}
