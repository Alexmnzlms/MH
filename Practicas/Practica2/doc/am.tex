% \chapter{Métodos de búsqueda}
\newpage
\section{Algoritmos meméticos}
%MAX 2 PAGS
%Descripción en pseudocódigo de la estructura del método de búsqueda y de todas aquellas operaciones relevantes de cada algoritmo. Este contenido, específico a cada algoritmo se detallará en los correspondientes guiones de prácticas. El pseudocódigo deberá forzosamente reflejar la implementación/ el desarrollo realizados y no ser una descripción genérica extraída de las transparencias de clase o de cualquier otra fuente. La descripción de cada algoritmo no deberá ocupar más de 2 páginas.
%Para esta primera práctica se incluirá la descripción en pseudocódigo del método de exploración del entorno, el operador de generación de vecino y la generación de soluciones aleatorias empleadas en el algoritmo de BL.
\subsection{Datos propios}
% En la búsqueda local es necesario controlar de alguna manera que vecinos se han generado y cuáles puede generar aún, por tanto, almaceno en un set el \textbf{vecindario} correspondiente a la solución que se está evaluando actualmente. El vecindario no es más que las posibles cambios que se pueden realizar en el vector solución, partiendo de un estado determinado.
% \begin{minted}
% [fontsize=\footnotesize, linenos]
% {cpp}
%    Pareja(int,int): set vecindario
% \end{minted}

Los algortimos meméticos se basan en la unión de los algoritmos genéticos con la bússqueda local, por lo que comparten los mismos datos propios.
\begin{minted}
[fontsize=\footnotesize, linenos]
{cpp}
   int: poblacion;
   int: ind_eval;
   int: matriz generacion;
   int: matriz seleccion;
   double: f_generacion;
   double: f_seleccion;
   int: mejor_generacion;
   double: f_mejor_generacion;
\end{minted}



\subsection{Descripción del algoritmo}

\subsection{Pseudocódigo del algoritmo}
\begin{minted}
[fontsize=\footnotesize, linenos]
{cpp}

\end{minted}

\subsection{Operadores propios}
\begin{itemize}
   \item Operador \textbf{mejor\_valor(sol,f\_sol,i)}:
   \begin{minted}
   [fontsize=\footnotesize, linenos]
   {cpp}

   \end{minted}
   \item Operador \textbf{busqueda\_local\_suave()}:
   \begin{minted}
   [fontsize=\footnotesize, linenos]
   {cpp}

   \end{minted}
\end{itemize}
