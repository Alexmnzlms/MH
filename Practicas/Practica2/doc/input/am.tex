% \chapter{Métodos de búsqueda}
\newpage
\section{Algoritmos meméticos}
%MAX 2 PAGS
%Descripción en pseudocódigo de la estructura del método de búsqueda y de todas aquellas operaciones relevantes de cada algoritmo. Este contenido, específico a cada algoritmo se detallará en los correspondientes guiones de prácticas. El pseudocódigo deberá forzosamente reflejar la implementación/ el desarrollo realizados y no ser una descripción genérica extraída de las transparencias de clase o de cualquier otra fuente. La descripción de cada algoritmo no deberá ocupar más de 2 páginas.
%Para esta primera práctica se incluirá la descripción en pseudocódigo del método de exploración del entorno, el operador de generación de vecino y la generación de soluciones aleatorias empleadas en el algoritmo de BL.
\subsection{Datos propios}
% En la búsqueda local es necesario controlar de alguna manera que vecinos se han generado y cuáles puede generar aún, por tanto, almaceno en un set el \textbf{vecindario} correspondiente a la solución que se está evaluando actualmente. El vecindario no es más que las posibles cambios que se pueden realizar en el vector solución, partiendo de un estado determinado.
% \begin{minted}
% [fontsize=\footnotesize, linenos]
% {cpp}
%    Pareja(int,int): set vecindario
% \end{minted}

Los algortimos meméticos se basan en la unión de los algoritmos genéticos con la búsqueda local, por lo que comparten los mismos datos propios.
\begin{minted}
[fontsize=\footnotesize, linenos]
{cpp}
   int: poblacion;
   int: ind_eval;
   int: matriz generacion;
   int: matriz seleccion;
   double: f_generacion;
   double: f_seleccion;
   int: mejor_generacion;
   double: f_mejor_generacion;
\end{minted}



\subsection{Descripción del algoritmo}
Los algoritmos memeticos implementados en esta practica, están construidos sobre un algoritmo genético, con esquema generacional y que utiliza el operador de cruce uniforme, es decir, \textbf{AGG\_UN}, puesto que es el algoritmo genético generacional que mejores resultados ha ofrecido durante las pruebas.
Se implementan 3 versiones de algoritmos meméticos:
\begin{itemize}
   \item \textbf{AM\_10-1.0}: Esta versión aplica una BLS cada 10 generaciones a todos los elementos de la población.
   \item \textbf{AM\_10-0.1}: Esta versión aplica una BLS cada 10 generaciones a 1 elemento de la población.
   \item \textbf{AM\_10-0.1mej}: Esta versión aplica una BLS cada 10 generaciones al mejor elemento de la población.
\end{itemize}
La Búsqueda Local Suave (\textbf{BLS}) es una variación de la búsqueda local que se centra en perder el mayor numero posible de evaluaciones. Existe un número máximo de fallos que puede cometer la BLS. La BLS recorre de menara aleatoria los genes de un cromosoma. Si puede mejorar el cromosoma, aplica el mejor cambio posible, y si no puede mejorarlo más, acumula un fallo. La BLS termina cuando se han recorrido todos los genes o cuando se ha superado el máximo de fallos.

\subsection{Pseudocódigo del algoritmo}
\begin{minted}
[fontsize=\footnotesize, linenos]
{cpp}
int: generacion = 0;

Generamos una poblacion inicial
//Igual que en la busqueda local.

Hacer:
   generacion++;
   Seleccionar el mejor de la población y guardarlo

   Aplicar seleccion

   Aplicar operador de cruce

   Aplicar elitismo y generacional

   Si generacion multiplo de 10:
      Aplicar BLS

Mientras evaluaciones menor que 100000
\end{minted}

\subsection{Operadores propios}
\begin{itemize}
   \item Operador \textbf{mejor\_valor(sol,f\_sol,i)}: Este operador recibe una solución, su valoración y el gen que se quiere mejorar. Si existe una majora para este gen, la realiza y devuelve verdadero. Si no, devuelve falso.
   \begin{minted}
   [fontsize=\footnotesize, linenos]
   {cpp}
   double: min = f_sol
   int: cluster_min
   int: vector sol_comp
   double: f_sol_comp
   int: matriz c

   Para j = 0 hasta tamaño de la solución:
      Añadr j a c[sol[j]]

   Para j = 0 hasta tamaño de la solución:
      sol_comp = sol
      Si sol[i] distinto de j Y tamaño de c[j] mayor que 1:
         sol_comp[i] = j
         f_sol_comp = evaluar_solucion(sol_comp)
         Si min > f_sol_comp:
            cluster_min = j
            min = f_sol_comp

   Si cluster_min no se ha asignado a nada:
      Devolver falso
   Si no:
      sol[i] = cluster_min
      f_sol = evaluar_solucion(sol)
      Devolver verdadero

   \end{minted}
   \item Operador \textbf{busqueda\_local\_suave()}: Aplica la búsqueda local suave a una solución determinada.
   \begin{minted}
   [fontsize=\footnotesize, linenos]
   {cpp}
   int: i = 0
   int: tam = tamaño solucion
   int: vector rsi = {0,1,2....tam-1}
   bool: mejora
   int: fallos
   int: umbral = 0.1*tam

   Ordenar rsi aleatoriamente

   mejora = verdadero
   Mientras mejora O fallos menor que umbral Y i < tam:
      Si mejor_valor(sol,f_sol,rsi[i])
         mejora = verdadero
      Si no:
         fallos++

      i++
   \end{minted}
\end{itemize}
