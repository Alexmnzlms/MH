\chapter{Experimentos y análisis de resultados}
\section{Semillas}
%Descripción de los casos del problema empleados y de los valores de los parámetros considerados en las ejecuciones de cada algoritmo (incluyendo las semillas utilizadas).
Para la ejecución de los algoritmos se han seleccionado las siguientes semillas, utilizando un algoritmo para probar que no producen ciclos en ninguna de las ejecuciones del algoritmo greedy para ninguno de los conjuntos de datos. Toda semilla que en menos de 1000 iteraciones del algoritmo greedy no obtenga resultado es rechazada. \\
Las semillas se han cambiado respecto a la práctica anterior debido al cambio de conjuntos de datos:
\begin{itemize}
   \item 2024614690
   \item 2024676296
   \item 2024677261
   \item 2024740484
   \item 2024740899
\end{itemize}


El código utilizado para encontrar semillas se encuentra en la función \textbf{buscar\_semilla()} que se incluye en el fichero main.cpp


\section{Análisis}
%Análisis de resultados. El análisis deberá estar orientado a justificar (según el comportamiento de cada algoritmo) los resultados obtenidos en lugar de realizar una mera “lectura” de las tablas. Se valorará la inclusión de otros elementos de comparación tales como gráficas de convergencia, boxplots, análisis comparativo de las soluciones obtenidas, representación gráfica de las soluciones, etc.
Este análisis se va a estructurar en 3 partes. En primer lugar comentaré los resultados de los data sets de \emph{Iris} y \emph{Rand}. Estos se comentan en un mismo apartado debido a la gran similitud que existe entre ellos. Posteriormente se analizarán los resultados del data set \emph{Ecoli}, que más complejidad y dificultad en el ajuste presenta, con clara diferencia. Por último se analizarán los resultados del nuevo data set, \emph{Newthyroid}, que si bien presenta la misma estructura de clasificación que \emph{Iris} y \emph{Rand}, supone un aumento en la dificultad de computo respecto a estos. Ya que tanto los data sets como las semillas se han modificado, se han repetido las ejecuciones para los algoritmos greedy y de búsqueda local.

\subsection{Iris \& Rand}
Los data sets de \emph{Iris} y \emph{Rand}, no presentan ninguna dificultad para los algoritmos genéticos y meméticos. Como podemos observar en la Tabla 6.10, todos los algortimos implementados en esta práctica consiguen converger a la solución óptima, tanto para \emph{Rand} como para \emph{Iris} con 10\% y 20\% de restricciones. Si bien es cierto, el algoritmo que peor ajuste realiza es el algoritmo greedy, pero como ya comentamos, el algoritmo greedy depende en gran medida de la colocación inicial de los centroides en el espacio de datos. Una buena colocación inicial nos ayuda a obtener una mejor solución.
\\
Estos conjuntos son perfectos para ajustar el funcionamiento de los algoritmos genéticos y meméticos. Sus cromosomas no poseen una longitud excesiva y solo cuenta con 3 clusters para la clasificación. Esto facilita que el límite de iteraciones (100000) sea más que suficiente para que las poblaciones de cromosomas converjan a una solución óptima.

\subsection{Ecoli}
El data set \emph{Ecoli} si que presenta un reto a la hora de obtener una solución optima. Con un número mayor de clusters y de genes por cromosoma, el límite de evaluaciones puede quedarse corto si estas no se optimizan de la mejor manera posible. Cuantas menos evaluaciones redundantes seamos capaces de eliminar, más generaciones podrá alcanzar el algortimo genético o memético y mejores soluciones aportará. Esta dificultad añadida al computo se puede notar en que el tiempo necesario para agotar las evaluaciones en el data set \emph{Ecoli} es 5 veces mayor que el necesario en los data sets \emph{Rand} e \emph{Iris}.
\\
Tomando como referencia ---tanto para 10\% de restricciones como para 20\%--- los algoritmos greedy y de búsqueda local, todos los algoritmos genéticos y meméticos mejoran la solución aportada por estos dos, siendo el algoritmo greedy el que mejores soluciones aporta. Esto se puede deber a una buena colocación de los centroides iniciales.
\\
También podemos apreciar que existen diferencias entre los algoritmos genéticos y meméticos. Estos últimos aportan mejores soluciones, lo que no es de extrañar, ya que a grandes rasgos se trata de un AGG\_UN al que se le aplica una BLS.
Entre el modelo generacional y el modelo estacionario, no podemos apreciar grandes diferencias en los resultados. Aunque si notar que para el modelo generacional, el operador que mejor resultado aporta es el operador de cruce uniforme, mientras que para el modelo estacionario, el operador que encuentra una mejor solución es el operador de cruce por segmento fijo.
Esto podría explicarse porque, en el modelo generacional, al reemplazar a todos los individuos en cada generación, se favorece la exploración del espacio de soluciones ---mientra que el operador de segmento fijo, al estar sesgado, favorece la exploración de las soluciones--- permitiendo encontrar mejores optimos. Para el modelo estacionario, al solo seleccionar a dos cromosomas cada generación, el operador de segmento fijo, explota a estos dos cromosomas, de manera que converjan a una mejor solución, en detrimento de una mayor exploración del espacio de cromosomas.
\\
Entre los algoritmos memeticos, encontramos que para un 10\% de restricciones, la mejor solucion la aporta AM\_10-0.1 y para un 20\% la aporta AM\_10-0.1 mej.
Estos dos meméticos, realizan la BLS a un 10\% de la población, por lo que podemos notar que eso es suficiente para encontrar buenas soluciones. AM\_10-0.1 es el algoritmo más costoso, debido a que realiza una mayor cantidad de BLS. A la vista de los resultados, tantas BLS no significan una mejora en los resultados. Si bien, AM\_10-1.0 mejora a los algoritmos genéticos en los que se basa, con aplicar la BLS a solo un 10\% es suficiente para encontrar las mismas o mejores soluciones.


\subsection{Newthyroid}
Por ultimo tenemos el data set \emph{Newthyroid}, que se encuntra en la interseccion entre \emph{Iris}, \emph{Rand} y \emph{Ecoli}. Posee el mismo número de cluster que \emph{Iris} y \emph{Rand}, pero su número de genes por cromosoma es mayor, acercandose a la dificultad de computo de \emph{Ecoli} y aportando una diferenciación en las soluciones obtenidas.
\\
En el caso de \emph{Newthyroid}, es necesario establecer una separación entre el 10\% de restricciones y el 20\% de restricciones.
\\
Para un 10\% de restricciones, el peor algoritmo es la búsqueda local, porque aunque parezca que algunos genéticos y meméticos aporten peores soluciones ---fijándonos en la infactibilidad obtenida---  el valor de la función objetivo que alcanza la búsqueda local es bastante mediocre. Por ello, todos los genéticos y meméticos mejoran esta solución.
\\
Curiosamente, en los algoritmos genéticos, observamos el caso contrario a \emph{Ecoli}. Los algoritmos AGG\_SF y AGE\_UN son los que aportan los mejores resultados respecto a los esquemas generacionales y estacionarios respectivamente. La solución que alcanzan estos algoritmos es de bastante calidad, superando a la aportada por la BL, exceptuando al algoritmo AGE\_SF, que no es capaz de superar a la búsqueda local.
\\
En el caso de los meméticos, alcanzan lo que podríamos llamar el optimo ---la mejor solución encontrada para \emph{Newthyroid} 10\%--- los algoritmos AM\_10-1.0 y AM\_10-0.1. Por su parte AM\_10-0.1mej, alcanza este óptimo solo en\% algunas iteraciones. De nuevo observamos que los algoritmos meméticos mejoran las soluciones aportadas por los genéticos.\\
\\
Para un 20\% de restricciones, el resultado obtenido por la búsqueda local, mejora bastante, superando al aportado por greedy, pero quedandose atrás si lo comparamos con los algoritmos genéticos y meméticos.
\\
En cuanto a los algoritmos geneticos, independientemente del esquema de reemplazo y selección, podemos ver que aquellos que utlizan el operador de cruce por segmento fijo, convergen al óptimo ---la mejor solución encontrada para \emph{Newthyroid} 20\%--- mientras que los que implementan operador de cruce uniforme no. En este caso vemos que la explotación que realiza el operador de segmento fijo, repercute positivamente en el resultado obtenido.
\\
Por parte de los meméticos, todos convergen a la solución óptima. Por lo que podemos ver que la BLS que se aplica en estos algoritmos, mejora las soluciones aportadas al utilizar el operador de cruce uniforme.

\subsection{Conclusión}
Como Conclusión, podemos ver que los algoritmos genéticos y meméticos son ampliamente superiores a los algoritmos de búsqueda local y greedy. Este resultado no es extraño, ya que, los algoritmos genéticos trabajan con poblaciones de varias decenas de soluciones a la vez, mientras que la BL o greedy buscan mejorar una única solución. La exploración que realizan los algoritmos basados en poblaciones es ampliamente superior a los que realizan los basados en trayectorias. Con el suficiente tiempo, los algoritmos genéticos y meméticos son capaces de encontrar soluciones más optimas, sin depender tanto del estado inicial. Esto, es en gran medida propiciado por el operador de mutación, que es el encargado de mejorar la diversidad de la población evitando que esta se estanque en un óptimo local y aumentado la exploración del espacio de soluciones.
\\
También vemos que al implementar una búsqueda local en combinación, con un algoritmo genético ---un memético--- en un menor número de generaciones se consiguen mejores soluciones. Es importante también equilibrar cuantas veces se realiza esta búsqueda local dentro de la población y a cuantos individuos se aplica. Ya que esta operación es bastante costosa y si bien, podríamos obtener soluciones muy buenas, esto sería en detrimento del tiempo.
