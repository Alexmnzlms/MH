% \chapter{Métodos de búsqueda}
\newpage
\section{Algoritmos genéticos}
%MAX 2 PAGS
%Descripción en pseudocódigo de la estructura del método de búsqueda y de todas aquellas operaciones relevantes de cada algoritmo. Este contenido, específico a cada algoritmo se detallará en los correspondientes guiones de prácticas. El pseudocódigo deberá forzosamente reflejar la implementación/ el desarrollo realizados y no ser una descripción genérica extraída de las transparencias de clase o de cualquier otra fuente. La descripción de cada algoritmo no deberá ocupar más de 2 páginas.
%Para esta primera práctica se incluirá la descripción en pseudocódigo del método de exploración del entorno, el operador de generación de vecino y la generación de soluciones aleatorias empleadas en el algoritmo de BL.
\subsection{Datos propios}
% En la búsqueda local es necesario controlar de alguna manera que vecinos se han generado y cuáles puede generar aún, por tanto, almaceno en un set el \textbf{vecindario} correspondiente a la solución que se está evaluando actualmente. El vecindario no es más que las posibles cambios que se pueden realizar en el vector solución, partiendo de un estado determinado.
% \begin{minted}
% [fontsize=\footnotesize, linenos]
% {cpp}
%    Pareja(int,int): set vecindario
% \end{minted}

En los algoritmos de búsquedas basados en poblaciones necesitamos almacenar un conjunto de soluciones, al que llamaremos población. A cada solución de la población la llamaremos cromosoma. Para cada iteración del algoritmos, diremos que se crea una nueva generación de cromosomas que sustituye a los cromosomas anteriores de la población. A cada posición dentro de un cromosoma se la denomina gen\\
Almacenamos en una variable el número de cromosomas que tiene una población. Además, almacenaremos cada generación de la población en una matriz. También debemos tener otra matriz para almacenar los cromosomas seleccionados en cada generación. Las evaluaciones las almacenaremos en una variable que almacena el numero de llamadas que se hace a la función que evalúa una solución.\\
Para ahorrar evaluaciones, a cada cromosoma, se le asocia un valor de su funcion objetivo que se calcula una sola vez. Estos valores se almacenan en vectores.
También se almacena en un vector la mejor solución encontrada hasta el momento y su valoración.
\begin{minted}
[fontsize=\footnotesize, linenos]
{cpp}
   int: poblacion;
   int: ind_eval;
   int: matriz generacion;
   int: matriz seleccion;
   double: f_generacion;
   double: f_seleccion;
   int: mejor_generacion;
   double: f_mejor_generacion;
\end{minted}

\subsection{Descripción del algoritmo}
Los algoritmos se nombran de la siguiente manera \textbf{AG[Esquema]\_[Cruce]}. Se han implementado las versiones \textbf{AGG\_UN}, \textbf{AGG\_SF}, \textbf{AGE\_UN}, \textbf{AGE\_SF}. Para facilitar la descripción, se describirán por separado los operadores de cruce de los esquemas de selección y reemplazo.

\subsubsection{Operadores de cruze}
Hay dos operadores de cruce:
\begin{itemize}
   \item \textbf{Operador de cruce uniforme [UN]}: El operador de cruce uniforme, se basa en escoger $ n/2 $ números aleatorios en el rango $[0,n-1]$. De uno de los dos padres, seleccionamos los genes que coincidan con estos números y los copiamos en la descendencia. Los genes que quedan por asignar se obtienen del segundo padre.
   \item \textbf{Operador de segmento fijo [SF]}: El operador de segmento fijo, se basa en generar dos números aleatorios \emph{r} y \emph{v} en el rango $[0,n-1]$. Seleccionamos uno de los padros y copiamos en la descendencia el rango de genes $ [r,((r+v)\,mod\,n) - 1] $. El rsto de genes se determinan de la misma manera que en el operador de cruce uniforme.\\
   Este operador esta sesgado. Si seleccionamos el segmento de datos del padre que mejor valor de la función objetivo tenga, estamos \textbf{favoreciendo la explotación} y si lo hacemos al revés, estaremos \textbf{favoreciendo la exploración}. En este caso, se favorece la \textbf{explotación}.
\end{itemize}
\newpage
\subsubsection{Esquemas de reemplazo y selección}
Hay dos esquemas de reemplazo y selección:
\begin{itemize}
   \item \textbf{Generacional [G]}: El esquema generacional funciona de la siguiente manera: seleccionamos tantos cromosomas como cromosomas tenga la población ---50 en nuestro caso---. La forma de selección se basa en realizar un torneo binario, es decir, enfrentar a dos cromosomas aleatorios y añadir a la selección el que mejor valor tenga en su función objetivo. Una vez tenemos la selección, generamos tantos cromosomas como cromosomas haya en nuestra selección, utilizando el operador de cruce que corresponda. En el esquema generacional hay un 70\% de probabilidad de cruce, así que aplicando esperanza matemática, solo el 70\% de las parejas cruzan, y el otro 30\% se mantiene igual. Una vez se han cruzado los cromosomas de la población, se aplica el operador de mutación. Cada gen tiene un 0.001\% de probabilidad de mutación. Si un gen muta, cambia aleatoriamente su valor a otro posible cluster. Cuando todos los cromosomas han mutado, se aplica el operador de reparación que arregla las posibles infactibilidades en las soluciones --- haber dejado un cluster vacío, por ejemplo---. Finalmente, se aplica el elitismo. El elitismo se basa en que, después de haber obtenido una nueva generación de cromosomas, se comprueba que la mejor solución de la generación anterior este presente en la nueva generación. Si no esta presente, se sustituye por el peor cromosoma de la generación.
   \item \textbf{Estacionario [E]}: El esquema estacionario funciona de forma similar al generacional. En cuanto a la selección, en el esquema estacionario solo se seleccionan 2 cromosomas de cada generación, y como es lógico, su probabilidad de cruce es 100\%. A esta selección , tambien se le aplica el operador de mutación y de reparación ---En las mismas condiciones que en el esquema generacional---. A la hora de devolver estos cromosomas a la población, se compara el mejor de la selección contra el peor de la población, y se añade a la población el mejor de los dos. Despues, se comparan el segundo mejor contra el segundo peor y se realiza el mismo proceso.
\end{itemize}
El criterio de parada del algortimo es no superar las 100000 evaluaciones.

\subsection{Pseudocódigo del algoritmo}
\begin{minted}
[fontsize=\footnotesize, linenos]
{cpp}
int: generacion = 0;

Generamos una poblacion inicial
//Igual que en la busqueda local.

Hacer:
   generacion++;
   Seleccionar el mejor de la población y guardarlo

   Aplicar seleccion

   Aplicar operador de cruce

   Si generacional:
      Aplicar elitismo y generacional
   Si estacionario:
      Aplicar estacionario

Mientras evaluaciones menor que 100000
\end{minted}

\newpage
\subsection{Operadores propios}
\begin{itemize}
   \item Operador \textbf{operador\_seleccion(t)}: Selecciona t elementos de la poblacion y los añade a la selección.
   \begin{minted}
   [fontsize=\footnotesize, linenos]
   {cpp}
   seleccionados = torneo_binario(t)
   //Seleccionamos t elementos mediante torneo binario

   Para i = 0 hasta fin de seleccionados:
      Añadir a seleccion el elemento i de poblacion
      Añadir a f_seleccion el elemento i de f_poblacion
   \end{minted}

   \item Operador \textbf{operador\_cruce\_uniforme(p1,p2)}: Aplica el operador de cruce uniforme a p1 y p2 para generar un nuevo vector solución.
   \begin{minted}
   [fontsize=\footnotesize, linenos]
   {cpp}
   int: genoma = tamaño p1 / 2
   int: gen
   int: vector genes

   Para i = 0 hasta genoma:
      gen = Numero aleatorio
      Si gen no esta en genes:
         Añadir gen a genes

   int: vector descendiente

   Para i = 0 hasta tamaño p1:
      Si i esta en genes:
         Añadir a descendiente gen i de p1
      Si no:
         Añadir a descendiente gen i de p2

      Devolver descendiente
   \end{minted}

   \item Operador \textbf{operador\_cruce\_segmento(p1,f\_p1,p2,f\_p2)}: Aplica el operador de cruce por segmento fijo a p1 y p2 para generar un nuevo vector solución. Las valoraciones de p1 y p2 se pasan como parámetro para ahorrar evaluaciones innecesarias.
   \begin{minted}
   [fontsize=\footnotesize, linenos]
   {cpp}
   int: tam = tamaño p1
   int: r = Numero aleatorio entre 0 y tam
   int: v = Numero aleatorio entre 0 y tam
   int: set genes_dominantes
   int: genoma = (tam - v) / 2
   int: gen
   int: set genes

   Para i = 0 hasta v:
      Añadir r+i%tam a genes_dominantes
   Para i = 0 hasta genoma:
      gen = Numero aleatorio entre 0 y tam
      Si gen no se encuentra en genes Y no se encuentra en genes dominantes:
         Añadir gen a genes

   int: vector mejor;
   int: vector peor;

   Escoger como mejor p1 o p2 segun f_p1 y f_p2
   Escoger como peor el otro

   int: vector descendiente
   Para i = 0 hasta tam:
      Si i se encuentra en genes_dominantes:
         Añadir a descendiente, gen i del mejor o peor padre,
         segun se favorezca la exploración o la explotación
      Si no:
         Si i se encuentra en genes:
            Añadir a descendiente, gen i de p1
         Si no:
            Añadir a descendiente, gen i de p2

   Devolver descendiente
   \end{minted}

\newpage
   \item Operador \textbf{reparar\_solucion(sol)}: Repara las infactibilidades presentes en un vector solución.
   \begin{minted}
   [fontsize=\footnotesize, linenos]
   {cpp}
   bool: reparacion = falso
   int: aleatorio
   int: tam_sol = tamaño sol - 1
   int matriz c
   int: vector c_vacio

   Para i = 0 hasta tamaño de sol:
      Añadir i a c[sol[i]]

   Para i = 0 hasta numero de cluster:
      Si c[i] esta vacio:
         reparacion = verdadero
         Añadir i a c_vacio


   Si reparacion:
      Recorrer el vector c_vacio:
         aleatorio = Numero aleatorio entre 0 y tam_sol
         Mientras haya al menos un elemento en c[sol[aleatorio]]:
            aleatorio = Numero aleatorio entre 0 y tam_sol

         Eliminar aleatorio de c[sol[aleatorio]]
         Actualizar sol[aleatorio] a elemento actual de c_vacio
         Añadir aleatorio a c[sol[aleatorio]]

   \end{minted}

   \item Operador \textbf{mutar\_solucion(sol)}: Aplica el operador de mutación a una solución.
   \begin{minted}
   [fontsize=\footnotesize, linenos]
   {cpp}
   int: aleatorio
   int: mutación
   int: p_mutacion = 1000

   Para i = 0 hasta tamaño de sol:
      mutacion = Numero aleatorio entre 1 y p_mutacion
      Si mutacion igual a 1:
         aleatorio = Numero aleatorio entre 0 y n_cluster
         sol[i] = aleatorio

   \end{minted}

   \item Operador \textbf{aplicar\_generacional()}: Aplica el esquema generacional a una población. Sustituimos la generación actual por la selección y conservamos el elitismo.
   \begin{minted}
   [fontsize=\footnotesize, linenos]
   {cpp}
   generacion = seleccion
   f_generacion = f_seleccion

   conservar_elitismo()

   limpiar seleccion
   limpiar f_seleccion
   \end{minted}

   \item Operador \textbf{conservar\_elitismo()}: Comprueba si la mejor solución generada en un generación se encuentra presente en la siguiente generación, y si no es así, la introduce en la generación.
   \begin{minted}
   [fontsize=\footnotesize, linenos]
   {cpp}
   int: sustituir

   Si mejor_generacion esta en seleccion:
      sustituir = seleccionar el peor de la generacion
      generacion[sustituir] = mejor_generacion;
      f_generacion[sustituir] = evaluar_solucion(mejor_generacion);

   \end{minted}

\newpage
   \item Operador \textbf{aplicar\_estacionario(n)}: Aplica el esquema estacionario a una población. Enfrenta al mejor seleccionado contra el peor de la generación iterativamente e introduce en la generación el mejor de ambos.
   \begin{minted}
   [fontsize=\footnotesize, linenos]
   {cpp}
   int: matriz estacion
   double: f_estacion
   int: matriz peores
   double: f_peores
   int: index_peor
   double: max, min
   int: i_max, i_min

   Para i = 0 hasta n:
      index_peor = seleccionar el peor de la generacion

      Añadir generacion[index_peor] a peores
      Añadir f_generacion[index_peor] a f_peores

      Borrar generacion[index_peor] de seleccion
      Borrar f_generacion[index_peor] de f_seleccion

   int: vector comprobado_peor
   int: vector comprobado_selec

   Para i = 0 hasta n:
      Para j = 0 hasta n:
         Si f_seleccion[j] menor que min Y j no esta en comprobado_selec:
            min = f_seleccion[j]
            i_min = j
      Para j = 0 hasta n:
         Si f_peores[j] mayor que max Y j no esta en comprobado_peor:
            max = f_peores[j]
            i_max = j

      Si f_seleccion[i_min] es menor que f_peores[i_max]:
         Añadir seleccion[i_min] a estacion
         Añadir f_seleccion[i_min] a f_estacion
      Si no:
         Añadir peores[i_min] a estacion
         Añadir f_peores[i_min] a f_estacion

      Añadir i_min a comprobado_selec
      Añadir i_max a comprobado_peor

   Para i = 0 hasta n:
      Añadir estacion[i] a generacion
      Añadir f_estacion[i] a f_generacion

   Limpiar seleccion
   Limpiar f_seleccion
   \end{minted}
\end{itemize}
