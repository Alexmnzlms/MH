\chapter{Experimentos y análisis de resultados}
\section{Semillas}
%Descripción de los casos del problema empleados y de los valores de los parámetros considerados en las ejecuciones de cada algoritmo (incluyendo las semillas utilizadas).
Para la ejecución de los algoritmos se han seleccionado las siguientes semillas, utilizando un algoritmo para probar que no producen ciclos en ninguna de las ejecuciones del algoritmo greedy para ninguno de los conjuntos de datos. Toda semilla que en menos de 1000 iteraciones del algoritmo greedy no obtenga resultado es rechazada. \\
Las semillas se han cambiado respecto a la práctica anterior debido al cambio de conjuntos de datos:
\begin{itemize}
   \item 2024614690
   \item 2024676296
   \item 2024677261
   \item 2024740484
   \item 2024740899
\end{itemize}


El código utilizado para encontrar semillas se encuentra en la función \textbf{buscar\_semilla()} que se incluye en el fichero main.cpp


\section{Análisis}
%Análisis de resultados. El análisis deberá estar orientado a justificar (según el comportamiento de cada algoritmo) los resultados obtenidos en lugar de realizar una mera “lectura” de las tablas. Se valorará la inclusión de otros elementos de comparación tales como gráficas de convergencia, boxplots, análisis comparativo de las soluciones obtenidas, representación gráfica de las soluciones, etc.
Este análisis se va a estructurar en 3 partes. En primer lugar comentaré los resultados de los conjuntos de Iris y Rand. Estos conjuntos se comentan juntos por la gran similitud que existe entre ellos. Posteriormente se analizarán los resultados del conjunto Ecoli, el conjunto que más complejidad y dificultad en el ajuste presenta, con clara diferencia. Por último se analizarán los resultados del nuevo conjunto de datos, Newthyroid, que si bien presenta la misma estructura de clasificación que Iris y Rand, supone un aumento en la dificultad de computo respecto a estos.

\subsection{Iris \& Rand}

\subsection{Ecoli}

\subsection{Newthyroid}
