\chapter{Experimentos y analisis de resultados}
\section{Semillas}
%Descripción de los casos del problema empleados y de los valores de los parámetros considerados en las ejecuciones de cada algoritmo (incluyendo las semillas utilizadas).
Para la ejecución de los algoritmos se han seleccionado las siguientes semillas, utilizando un algoritmo para probar que no producen ciclos en ninguna de las ejecuciones del algoritmo greedy para ninguno de los conjuntos de datos. Toda semilla que en menos de 1000 iteraciones del algoritmo greedy no obtenga resultado es rechazada. \\
La búsqueda de las semillas ha sido relativamente compleja ya que la mayoria de las primeras semillas que probé no obtenían resultados para el data set \emph{Ecoli}.
\begin{itemize}
   \item
   \item
   \item
   \item
   \item
\end{itemize}

El código utilizado para encontrar semillas se encuentra en la función \textbf{buscar\_semilla()} que se incluye en el fichero main.cpp


\section{Análisis}
%Análisis de resultados. El análisis deberá estar orientado a justificar (según el comportamiento de cada algoritmo) los resultados obtenidos en lugar de realizar una mera “lectura” de las tablas. Se valorará la inclusión de otros elementos de comparación tales como gráficas de convergencia, boxplots, análisis comparativo de las soluciones obtenidas, representación gráfica de las soluciones, etc.
