\newpage
\section{Enfriamiento Simulado}
\subsection{Descripción del algoritmo}
El enfriamiento simulado imita el proceso de enfriamiento físico de un sistema, de manera que cuando la temperatura es alta el sistema se encuentra excitado y por tanto se potencia la diversificación y según decrece la temperatura, la diversificación da paso la exploración de la mejor solución encontrada. \\
Primero se calcula la temperatura inicial de la siguiente manera: $ \frac{\mu \cdot F_{objetivo}}{-1log(\phi)} $. \\
El proceso que rige como la temperatura decrece se conoce como el esquema de enfriamiento y en esta implementación del algoritmo se adapta un esquema proporcional ($t_{k+1} = \alpha \cdot t_{k}$).\\
Llamaremos $L(T)$ al bucle de enfriamiento interno, es decir, el bucle en el que se generan y aceptaran soluciones en función de la temperatura actual. $L(T)$ finaliza cunado el numero de vecinos generados supera un numero máximo o el numero de soluciones aceptadas supera un umbral.
En nuestro caso el numero máximo de vecinos es 10 veces el tamaño de la solución y el numero de soluciones aceptadas ses un 10\% del numero máximo de vecinos.\\
Existen dos condiciones de aceptación de una solución:
\begin{itemize}
   \item El valor objetivo de la solución generada es inferior al valor de la función objetivo de la ultima solución aceptada.
   \item Se genera un numero aleatorio entre 0 y 1. Si este numero es inferior a $e^{\frac{-diferencia}{t_{k}}}$ la nueva solución se acepta independientemente del valor de su función objetivo. Esto se conoce como el criterio de Metropolis.
\end{itemize}
El criterio de Metropolis nos permite combinar la diversificación de la solución cuando la temperatura es alta, dado que siempre se cumplirá la condición del criterio de Metropolis (para temperaturas muy altas es de interés notar que el criterio de Metropolis es superior a 1), con la explotación de la mejor solución. A medida que la temperatura actual decrece, esta condición reduce drásticamente su valor, permitiendo la explotación de la solución unicamente aceptando una solución cuando mejora el valor de la función objetivo de la solución actual.
Cuando se acepta una solución, se toma como solución de partida a la hora de generar vecinos nuevos. También se almacena cual es la mejor solución encontrada hasta el momento, siento esta la solución final del algoritmo.
La condición de para del algoritmo es que supere 100000 iteraciones, que la temperatura actual $t_{k} <= 0.001$ o que exista un $L(T)$ en el que no se haya aceptado ninguna solución.
Se utiliza un esquema de enfriamiento proporcional debido a que, durante la realización de esta práctica, se hizo notable que el enfriamiento de cauchy modificado que se proporciona, reducía drásticamente la temperatura en las primeras iteraciones, impidiendo así una correcta diversificación de la solución.
\subsection{Pseudocódigo del algoritmo}
\begin{minted}
[fontsize=\footnotesize, linenos]
{cpp}

int: t_inicial, t_final, t_actual, vecinos, exitos, eval
double: f_objetivo_acp, f_objetivo_mejor
int: vector[] solucion_ant, solucion_mejor

solucion_inicial()

t_inical = (mu * f_objetivo) / (-1 * log(phi))
t_final = 0.01
t_actual = t_inical

Mientras eval < max_eval Y t_actual > t_final Y exitos != 0:
   vecinos = 0
   exitos = 0

   Mientras vecinos < max_vecinos Y exitos < max_exitos:
      generar_vecino_es()
      vecinos++
      eval++
      diferencia = f_objetivo - f_objetivo_acp;
      uniforme = numero aleatorio entre 0 y 1

      Si diferencia < 0 O uniforme <= exp((-diferencia) / t_actual):
         exitos++
         solucion_acp = solucion
         f_objetivo_acp = f_objetivo

         Si f_objetivo < f_objetivo_mejor:
            solucion_mejor = solucion
            f_objetivo_mejor = f_objetivo

      Si no:
         solucion = solucion_acp

   solucion = solucion_acp
   f_objetivo = f_objetivo_acp
   t_actual = 0.9*t_actual

solucion = solucion_mejor
f_objetivo = f_objetivo_mejor

\end{minted}

\subsection{Operadores propios}
\begin{itemize}
   \item Operador \textbf{generar\_vecino\_es()}: Mismo planteamiento que el operador vecino de la búsqueda local, pero como en ES no es necesario controlar que no se repita un vecino, ya no es necesario implementar la estructura del vecindario.
   \begin{minted}
   [fontsize=\footnotesize, linenos]
   {cpp}
   bool: salir = falso
   int: pos, n, c

   Mientras no salir Y quedan vecinos que explorar:
      salir = falso
      pos = número aleatorio 0...i
      n = número aleatorio 0...k

      Si el tamaño del cluster de solucion[pos] - 1 > 0:
         salir = verdadero
         Borrar Pareja (pos,n) de vecindario
         c = solucion[pos]
         Marcar solucion[pos] como invalido

         Para cada cluster j en el conjunto de cluster:
            Vaciar j

         Para cada elemento e de la solucion:
            Si e es valido:
               Introducir e en el cluster solucion[e]

         infactibilidad -= restricciones_incumplidas(pos,c)
         infactibilidad += restricciones_incumplidas(pos,n)
         solucion[pos] = n
         Añadir pos al cluster n
         desviacion_general()
         f_objetivo = desv_gen + (infactibilidad*lambda)
   \end{minted}
\end{itemize}
