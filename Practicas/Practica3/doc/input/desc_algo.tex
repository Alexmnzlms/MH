\chapter{Descripción de los algoritmos empleados}
%MAX 4 PAGS
%Breve descripción de la aplicación de los algoritmos empleados al problema (máximo 4 páginas): Todas las consideraciones comunes a los distintos algoritmos se describirán en este apartado, que será previo a la descripción de los algoritmos específicos. Incluirá por ejemplo la descripción del esquema de representación de soluciones y la descripción en pseudocódigo (no código) de la función objetivo y los operadores comunes.
\section{Función objetivo}

Todos los algoritmos implementados comparten la misma función objetivo: $ F_{objetivo} = \bar{C} + (infeasibility \cdot \lambda) $\\
\\
El valor esta función se calcula como la suma de la \textbf{desviación general} ($ \bar{C} $) y la infactibilidad de la partición (\emph{infeasibility}) escalada por un parámetro ($ \lambda $). Este parámetro tiene la función de dar relevancia al término \emph{infeasibility}. Si se establece correctamente, primero da prioridad a reducir las restricciones incumplidas por encima de la desviación general.\\ Este parámetro se calcula como: $ \lambda = \frac{\left \lceil D \right \rceil}{\left | R \right |}$. Siendo D la distancia entre los dos puntos más lejanos y R el número total de restricciones. \\
El objetivo de los algoritmos descritos a continuación es minimizar el valor de esta función objetivo.

\begin{itemize}
   \item \textbf{K-medias Restringido Débil}: Algoritmo de heurística greedy que se basa en asignar un elemento al cluster de cuyo centroide se encuentre más cercano de manera iterativa, hasta obtener el mejor ajuste posible.
   \item \textbf{Búsqueda Local}: Algoritmo de búsqueda basado en trayectorias, se vale de la exploración de vecindarios para obtener la mejor solución posible.
   \item \textbf{Algoritmos Genéticos}: Algoritmos de búsqueda basado en poblaciones. En esta práctica implementamos un esquema generacional y uno estacionario, y los combinamos con dos operadores de cruce, uno uniforme y otro de segmento fijo, obteniendo así 4 algoritmos distintos.
   \item \textbf{Algoritmos Meméticos}: Algoritmos de búsqueda basado en poblaciones. Se basa en la implementación de un algortimo genético con un esquema generacional, con la diferencia de que cada determinado número de generaciones, se aplica una búsqueda local a un determinado número de cromosomas. En esta práctica se implementan 3 versiones que se desarrollarán más adelante.
   \item \textbf{Búsqueda Multiarranque Básica}: Algoritmo de búsqueda basado en trayectorias. Se basa en realizar una búsqueda local partiendo de soluciones aleatorias distintas (10 BL con 10000 iteraciones) y guardar la solución cuyo valor de función objetivo es minima. La solución que aporta la BMB es la solución guardada, es decir, la minima.
   \item \textbf{Enfriamiento Simulado}: Algoritmo de búsqueda basado en trayectorias. Este algoritmo trata de replicar el proceso de enfriamiento de un sistema fisico. El proceso de busqueda lleva asociado una temperatura. Cuando la temperatura es alta, se aceptan soluciones con peor valor de la función objetivo y segun va disminuyendo la temperatura, solo se aceptan soluciones mejores.
   \item \textbf{Iterated Local Search}: Algortimo de búsqueda basado en trayectorias. Similar a la Búsqueda Multiarranque Básica, con la diferencia de que a la solución de una BL se le aplica un operador de mutación y se ejecuta la siguiente BL comenzando en esta solución.
   \item \textbf{Iterated Local Search - Enfriamiento Simulado}: Algoritmo de búsqueda basado en trayectorias. Mismo algoritmo que la ILS pero se aplica el enfriamiento simulado en lugar de la búsqueda local.

\end{itemize}
\newpage
\section{Representación de los datos}
Los datos comunes a ambos algoritmos se representan de la siguiente manera:

\begin{itemize}
   \item Para los \textbf{datos} utilizo una matriz \emph{posiciones} de números reales de dimensión $ i \times n $.
   \begin{minted}
   [fontsize=\footnotesize, linenos]
   {cpp}
      double: matriz[i][n] posiciones
   \end{minted}

   \item Para los \textbf{centroides} de cada cluster utilizo una matriz \emph{centroides} de dimensión $ k \times n $.
   \begin{minted}
   [fontsize=\footnotesize, linenos]
   {cpp}
      double: matriz[k][n] centroides
   \end{minted}

   \item Para las \textbf{restricciones}, he elegido no utilizar las representación en forma de matriz, porque es muy costoso recorrerla secuencialmente, y la representación en forma de lista no te permite acceder a un elemento en concreto, por eso he decidido utilizar un map \emph{restricciones}. Este tipo de estructura se puede recorrer secuencialmete de forma eficiente y ademas, existe un operador de búsqueda para poder acceder a un elemento concreto. El map se compone de dos elementos: la clave y el valor. La clave es la pareja de elementos $ x_{i}, x_{j}$ y el valor es 1 si la restricción es de tipo \textbf{ML} o -1 si es de tipo \textbf{CL}.
   \begin{minted}
   [fontsize=\footnotesize, linenos]
   {cpp}
      Pareja(int,int),int: map restricciones
   \end{minted}

   \item Los elementos $ x_{i} $ que pertenecen a los distintos \textbf{clusters} los almaceno en una matriz \emph{clusters} de dimensión $ k \times i $.
   \begin{minted}
   [fontsize=\footnotesize, linenos]
   {cpp}
      double: matriz[k][i] clusters
   \end{minted}

   \item Finalmente la \textbf{partición} \emph{C} la represento en un vector de enteros \emph{solucion}, en los que posición del vector \emph{i} indica el elemento $ x_{i} $ y el contenido de la posición \emph{i}, \emph{solucion[i]} indica el cluster $ c_{i} $ al que pertenece.
   \begin{minted}
   [fontsize=\footnotesize, linenos]
   {cpp}
      int: vector[i] solucion
   \end{minted}
\end{itemize}


\section{Operadores comunes}
Hay una serie de operadores que son comunes a los dos algoritmos, los describo a continuación:
\begin{itemize}
   \item Operador \textbf{calcular\_centroide(cluster i)} Calcula el centroide de un cluster i:
   \begin{minted}
   [fontsize=\footnotesize, linenos]
   {cpp}
   Para cada característica u del centroide i:
      u = 0

   Para cada elemento j del cluster i:
      Para cada característica c, u del elemento j, centroide i:
         u += 1/k * c
   \end{minted}
   \item Operador \textbf{distancia\_intracluster(cluster i)}: Calcula la distancia intracluster de un cluster i.
   \begin{minted}
   [fontsize=\footnotesize, linenos]
   {cpp}
   Para todos los clusters:
      d_intracluster = 0

   Para cada elemento j del cluster i:
      Para cada característica c, u del elemento j, centroide i :
         d_intracluster += 1/k * abs(c - u) * abs(c - u)
   \end{minted}
   \item Operador \textbf{desviación\_general()}: Calcula la desviación general del problema.
   \begin{minted}
   [fontsize=\footnotesize, linenos]
   {cpp}
   desv_gen = 0

   Para cada distancia_intracluster i del cluster i:
      desv_gen += 1/k * i
   \end{minted}
   \item Operador \textbf{calcular\_lambda()}: Calcula el parámetro $ \lambda $
   \begin{minted}
   [fontsize=\footnotesize, linenos]
   {cpp}
   lambda = 0
   double: d, d_max = 0
   int: cluster

   Para cada elemento e en posiciones:
      cluster = Cluster al que pertenece e
      Para cada número n de 0...k:
         Si n != cluster:
            Para cada elemento c de cluster n:
               d = Distancia entre n y e
               Si d es mayor que d_max:
                  d_max = d

   lambda = d_max / restricciones

   \end{minted}

   \item Operador \textbf{calcular\_infact\_sol(solucion)}: Calcula la infacibilidad de un vector solución dado
   \begin{minted}
   [fontsize=\footnotesize, linenos]
   {cpp}
   int: infactibilidad = 0
   Para cada elemento i de la solucion:
      Desde j = i + 1 hasta el fin de solución:
         Si la pareja (i,j) pertence al set de restricciones:
            Si j igual -1 Y elemento i igual elemento j de sol:
               infactibilidad++
            Si j igual 1 Y elemento i distinto elemento j de sol:
               infactibilidad++

   Devolver infactibilidad
   \end{minted}

\end{itemize}
