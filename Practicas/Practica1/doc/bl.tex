\newpage
\section{Métodos de búsqueda}
\subsection{Búsqueda Local}
%MAX 2 PAGS
%Descripción en pseudocódigo de la estructura del método de búsqueda y de todas aquellas operaciones relevantes de cada algoritmo. Este contenido, específico a cada algoritmo se detallará en los correspondientes guiones de prácticas. El pseudocódigo deberá forzosamente reflejar la implementación/ el desarrollo realizados y no ser una descripción genérica extraída de las transparencias de clase o de cualquier otra fuente. La descripción de cada algoritmo no deberá ocupar más de 2 páginas.
%Para esta primera práctica se incluirá la descripción en pseudocódigo del método de exploración del entorno, el operador de generación de vecino y la generación de soluciones aleatorias empleadas en el algoritmo de BL.
\subsubsection{Pseudocódigo del algoritmo}
\begin{minted}
[fontsize=\footnotesize, linenos]
{cpp}
   double: f_objetivo_ant, infactibilidad_ant
   int: vector[] solucion_ant
   int: i

   solucion_inicial()
   generar_vecindario()

   f_objetivo_ant = f_objetivo
   solucion_ant = solucion
   infactibilidad_ant = infactibilidad

   do:
      generar_vecino()
      i++

      Si nueva f_objetivo es menor que f_objetivo_ant:
         f_objetivo_ant = f_objetivo
         solucion_ant = solucion
         infactibilidad_ant = infactibilidad
         generar_vecindario()

      Si no:
         solucion = solucion_ant
         infactibilidad = infactibilidad_ant

      Si no quedan vecinos que generar:
         leer_vecino()

   while: i menor que 100000 Y quedan vecinos que explorar
\end{minted}

\subsubsection{Operadores propios}
\begin{itemize}
   \item Operador \textbf{solucion\_inicial()}:
   \begin{minted}
   [fontsize=\footnotesize, linenos]
   {cpp}
   \end{minted}
   \item Operador \textbf{leer\_solucion()}:
   \begin{minted}
   [fontsize=\footnotesize, linenos]
   {cpp}
   \end{minted}
   \item Operador \textbf{calcular\_lambda()}:
   \begin{minted}
   [fontsize=\footnotesize, linenos]
   {cpp}
   \end{minted}
   \item Operador \textbf{distancia\_nodo\_nodo()}:
   \begin{minted}
   [fontsize=\footnotesize, linenos]
   {cpp}
   \end{minted}
   \item Operador \textbf{generar\_vecindario()}:
   \begin{minted}
   [fontsize=\footnotesize, linenos]
   {cpp}
   \end{minted}
   \item Operador \textbf{buscar\_cluster(elemento n)}:
   \begin{minted}
   [fontsize=\footnotesize, linenos]
   {cpp}
   \end{minted}
   \item Operador \textbf{generar\_vecino()}:
   \begin{minted}
   [fontsize=\footnotesize, linenos]
   {cpp}
   \end{minted}
   \item Operador \textbf{leer\_vecino()}:
   \begin{minted}
   [fontsize=\footnotesize, linenos]
   {cpp}
   \end{minted}

\end{itemize}
