\newpage
\section{Métodos de búsqueda}
\subsection{Búsqueda Local}
%MAX 2 PAGS
%Descripción en pseudocódigo de la estructura del método de búsqueda y de todas aquellas operaciones relevantes de cada algoritmo. Este contenido, específico a cada algoritmo se detallará en los correspondientes guiones de prácticas. El pseudocódigo deberá forzosamente reflejar la implementación/ el desarrollo realizados y no ser una descripción genérica extraída de las transparencias de clase o de cualquier otra fuente. La descripción de cada algoritmo no deberá ocupar más de 2 páginas.
%Para esta primera práctica se incluirá la descripción en pseudocódigo del método de exploración del entorno, el operador de generación de vecino y la generación de soluciones aleatorias empleadas en el algoritmo de BL.
\subsubsection{Datos propios}


\subsubsection{Pseudocódigo del algoritmo}
\begin{minted}
[fontsize=\footnotesize, linenos]
{cpp}
double: f_objetivo_ant, infactibilidad_ant
int: vector[] solucion_ant
int: i

solucion_inicial()
generar_vecindario()

f_objetivo_ant = f_objetivo
solucion_ant = solucion
infactibilidad_ant = infactibilidad

Hacer:
   generar_vecino()
   i++

   Si nueva f_objetivo es menor que f_objetivo_ant:
      f_objetivo_ant = f_objetivo
      solucion_ant = solucion
      infactibilidad_ant = infactibilidad
      generar_vecindario()

   Si no:
      solucion = solucion_ant
      infactibilidad = infactibilidad_ant

   Si no quedan vecinos que generar:
      leer_vecino()

Mientras: i menor que 100000 Y quedan vecinos que explorar
\end{minted}

\subsubsection{Operadores propios}
\begin{itemize}
   \item Operador \textbf{solucion\_inicial()}:
   \begin{minted}
   [fontsize=\footnotesize, linenos]
   {cpp}
   int: vector[] index
   int: matriz[][] c

   Para numero n de 0...i:
      Añadir n a index

   Ordenar index aleatoriamente

   Para cada fila j en c:
      Añadir a j elemento de index
      Pasar al siguiente elemento

   Para los elementos restantes en index:
      Añadir a fila aleatoria 0...k de c elemento de index
      Pasar al siguiente elemento

   Para cada numero n de 0...k:
      Para los elementos e de la fila n de c:
         Añadir n a solucion en la posicion c[n][e]

   Para cada numero n de 0...k:
      calcular_centroide(n)
   \end{minted}
   \item Operador \textbf{calcular\_lambda()}:
   \begin{minted}
   [fontsize=\footnotesize, linenos]
   {cpp}
   lambda = 0
   double: d, d_max = 0
   int: cluster

   Para cada elemento e en posiciones:
      cluster = Cluster al que pertenece e
      Para cada numero n de 0...k:
         Si n != cluster:
            Para cada elemento c de cluster n:
               d = Distancia entre n y e
               Si d es mayor que d_max:
                  d_max = d

   lambda = d_max / restricciones

   \end{minted}
   \item Operador \textbf{generar\_vecindario()}:
   \begin{minted}
   [fontsize=\footnotesize, linenos]
   {cpp}
   Vaciar vecindario

   Para cada elemento e en posiciones:
      Para cada numero n de 0...k:
         Si n es disitinto al cluster al que pertenece e
         Y en el cluster al que pertenece e hay al menos 1 elemento:
            Insertar en vecindario Pareja (e,n)
   \end{minted}
   \item Operador \textbf{generar\_vecino()}:
   \begin{minted}
   [fontsize=\footnotesize, linenos]
   {cpp}
   bool: salir = falso
   int: pos, n, c

   Mientras no salir Y quedan vecinos que explorar:
      salir = falso
      pos = numero aleatorio 0...i
      n = numero aleatorio 0...k
      Buscar Pareja (pos,n) en vecindario

      Si Pareja (pos,n) existe en vecindario:
         salir = verdadero
         Borrar Pareja (pos,n) de vecindario
         c = solucion[pos]
         Marcar solucion[pos] como invalido

         Para cada cluster j en el conjunto de cluster:
            Vaciar j

         Para cada elemento e de la solucion:
            Si e es valido:
               Introducir e en el cluster solucion[e]

         infactibilidad -= restricciones_incumplidas(pos,c)
         infactibilidad += restricciones_incumplidas(pos,n)
         solucion[pos] = n
         Añadir pos al cluster n
         desviacion_general()
         f_objetivo = desv_gen + (infactibilidad*lambda)
   \end{minted}

\end{itemize}
