\chapter{Técnicas de Búsqueda Local y Algoritmos Greedy para el Problema del Agrupamiento con Restricciones}

\section{Descripción del problema}
%MAX 1 PAGS
El problema elegido es el \textbf{Problema del Agrupamiento con Restricciones}, a partir de ahora \textbf{PAR}, es una variante del problema de agrupamiento clásico.
El problema de agrupamiento clásico consiste en que dado un conjunto $ X $ de datos con \emph{n} caracteristicas, hay que encontrar una particion \emph{C} de tal manera que se minimice la desviación general de cada $ c_{i} \in C $.\\
En la variante \textbf{PAR}, se introduce el concepto de restricción. Nosotros utilizamos las restricciones de instancia que pueden ser dos tipos:
\begin{itemize}
   \item Restricciones \textbf{ML}(\emph{Must-link}): Dos elementos $ x_{i} \in X $ que posean una resticcion ML, deben pertenecer al mismo $ c_{i} \in C $.
   \item Restricciones \textbf{CL}(\emph{Cannot-link}): Dos elementos $ x_{i} \in X $ que posean una resticcion CL, deben pertenecer a $ c_{i} \in C $ distintos.
\end{itemize}
Además, estas restricciones serán débiles, es decir, el objetivo es minimizar tanto el número de restricciones incumplidas --- una solucion factible puede incumplir restricciones --- como la desviación general de cada $ c_{i} \in C $.
\subsection{Formalización de los datos}
\begin{itemize}
   \item Los \textbf{datos} se repretan en una matriz $ i \times n $ siendo \emph{i} el numero de datos que tenemos y \emph{n} el numero de caracteristicas que tiene cada $ x_{i} \in X $ \\
   $ \vec{x_{i}} = \{x_{i0}\ ...\ x_{in}\} $ donde cada $ x_{ij} \in \mathbb{R} $
   \item Una \textbf{particion} \emph{C} consiste en un conjunto de k clusters. $ C = \{c_{0}\ ...\ c_{k}\} $. Cada $ c_{i} $ contiene un conjunto de elementos $ x_{i} $. El numero de elementos de $ c_{i} $ es $ \left | c_{i} \right | $ y normalmente un cluster $ c_{i} $ tiene asociada una etiqueta $ l_{i} $ --- Esto no lo utilizaremos en la implementación del problema ---.
   \item Para cada cluster $ c_{i} $ se puede calcular su \textbf{centroide} $ \vec{\mu_{i}} $ como el promedio de los elementos $ x_{i} \in c_{i} $.\\
   $ \vec{\mu_{i}} = \frac{1}{\left | c_{i} \right |}\sum_{\vec{x_{j}} \in c_{i}} \vec{x_{j}} $.
   \item La \textbf{distancia media intra-cluster} $ \bar{c_{i}} $ se define como la media de las ditancias de cada $ x_{i} \in c_{i} $ a su centroide $ \mu_{i} $.\\
   $ \bar{c_{i}} = \frac{1}{\left | c_{i} \right |}\sum_{\vec{x_{j}} \in c_{i}} \left \| \vec{x_{j}} - \vec{\mu_{i}} \right \|_{2} $
   \item La \textbf{desviación general} de la partición \emph{C} se calcula como la media de las distancias medias intra-\\cluster $ \bar{c_{i}} $.\\
   $ \bar{C} = \frac{1}{k}\sum_{c_{i} \in C} \bar{c_i} $.
   \item El \textbf{conjunto de restricciones totales} \emph{R} se calcula como la union entre el conjunto de restricciones \emph{ML} y el conjunto de restricciones \emph{CL}.\\
   $ \left | R \right | $ es el numero de restricciones total $ \left | R \right | = \left | ML \right | + \left | Cl \right | $.
   \item La \textbf{infactibilidad} --- \emph{infeasibility} --- se calcula como el numero de restricciones que incumple una partición \emph{C} del conjunto \emph{X} dado un conunto de restricciones \emph{R}. Se define $ V(\vec{x_{i}},\vec{x_{j}}) $ como una funcion que devuelve 1 si la pareja $ (\vec{x_{i}},\vec{x_{j}}) $ incumple alguna restricción.\\
   $ infeasibility = \sum_{i=0}^{n} \sum_{j=i+1}^{n} V(\vec{x_{i}},\vec{x_{j}}) $

\end{itemize}

\section{Descripción de los algoritmos empleados}
%MAX 4 PAGS
%Breve descripción de la aplicación de los algoritmos empleados al problema (máximo 4 páginas): Todas las consideraciones comunes a los distintos algoritmos se describirán en este apartado, que será previo a la descripción de los algoritmos específicos. Incluirá por ejemplo la descripción del esquema de representación de soluciones y la descripción en pseudocódigo (no código) de la función objetivo y los operadores comunes.

En esta práctica se han implemtado dos algoritmos:
\begin{itemize}
   \item \textbf{K-medias Restringido Débil}: Algoritmo de heurística greedy, que busca minimizar tanto la desviación general como la \emph{infeasibility}. $ F_{objetivo} = \bar{C} + infeasibility $
   \item \textbf{Búsqueda Local}: Como método de búsqueda por trayectorias utilizaremos la búsqueda local, en este caso, se busca minimizar la desviación general y la \emph{infeasibility} multiplicado por un parametro $ \lambda $ que se describira mas adelante. $ F_{objetivo} = \bar{C} + (infeasibility \cdot \lambda) $
\end{itemize}

\subsection{Representación de los datos}
Los datos comunes a ambos algoritmos se repesentan de la siguiente manera:

\begin{itemize}
   \item Para los \textbf{datos} utilizo una matriz \emph{posiciones} de numeros reales de dimensión $ i \times n $.
   \begin{minted}
   [fontsize=\footnotesize, linenos]
   {cpp}
      double: matriz[i][n] posiciones
   \end{minted}
\end{itemize}


\begin{itemize}
   \item Para los \textbf{centroides} de cada cluster utilizo una matriz \emph{centroides} de dimensión $ k \times n $.
   \begin{minted}
   [fontsize=\footnotesize, linenos]
   {cpp}
      double: matriz[k][n] centroides
   \end{minted}
\end{itemize}


\begin{itemize}
   \item Para las \textbf{restricciones}, he elegido no utlizar las representación en forma de matriz, porque es muy costoso recorrerla secuencialmente, y la representación en forma de lista no te permite acceder a un elemento en concreto, por eso he decidido utilizar un map \emph{restricciones}. Este tipo de estructura se puede recorrer secuencialmete de forma eficiente y ademas, existe un operador de búsqueda para poder acceder a un elemento concreto. El map se compone de dos elementos: la clave y el valor. La clave es la pareja de elementos $ x_{i}, x_{j}$ y el valor es 1 si la resticcion es de tipo \textbf{ML} o -1 si es de tipo \textbf{CL}.
   \begin{minted}
   [fontsize=\footnotesize, linenos]
   {cpp}
      Pareja(int,int),int: map restricciones
   \end{minted}
\end{itemize}


\begin{itemize}
   \item Los elementos $ x_{i} $ que pertenecen a los distintos \textbf{clusters} los almaceno en una matriz \emph{clusters} de dimensión $ k \times i $.
   \begin{minted}
   [fontsize=\footnotesize, linenos]
   {cpp}
      double: matriz[k][i] clusters
   \end{minted}
\end{itemize}


\begin{itemize}
   \item Finalmente la \textbf{particion} \emph{C} la represnto en un vector de enteros \emph{solucion}, en los que posicion del vector \emph{i} indica el elemento $ x_{i} $ y el contenido de la posicion \emph{i}, \emph{solucion[i]} indica el cluster $ c_{i} $ al que pertenece.
   \begin{minted}
   [fontsize=\footnotesize, linenos]
   {cpp}
      int: vector[i] solucion
   \end{minted}
\end{itemize}


\subsection{Operadores comunes}
Hay una serie de operadores que son comunes a los dos algoritmos, los describo a continuación:
\begin{itemize}
   \item Operador \textbf{calcular\_centroide(cluster i)}:
   \begin{minted}
   [fontsize=\footnotesize, linenos]
   {cpp}
      Para cada caracteristica u del centroide i:
         u = 0

      Para cada elemento j del cluster i:
         Para cada caracteristica c, u del elemento j, centroide i:
            u += 1/k * c
   \end{minted}
   \item Operador \textbf{distancia\_intracluster(cluster i)}:
   \begin{minted}
   [fontsize=\footnotesize, linenos]
   {cpp}
      Para todos los clusters:
         d_intracluster = 0

      Para cada elemento j del cluster i:
         Para cada caracteristica c, u del elemento j, centroide i :
            d_intracluster += 1/k * abs(c - u) * abs(c - u)
   \end{minted}
   \newpage
   \item Operador \textbf{desviación\_general()}:
   \begin{minted}
   [fontsize=\footnotesize, linenos]
   {cpp}
      desv_gen = 0

      Para cada distancia_intracluster i del cluster i:
         desv_gen += 1/k * i
   \end{minted}

   \item Operador \textbf{limpiar\_clusters()}:
   \begin{minted}
   [fontsize=\footnotesize, linenos]
   {cpp}
      Para cada cluster i:
         Eliminar elementos de i

      infactibilidad = 0
   \end{minted}
   \item Operador \textbf{restricciones\_incumplidas(elemento n, cluster c)}:
   \begin{minted}
   [fontsize=\footnotesize, linenos]
   {cpp}
      incumplidas = 0

      Para cada elemento i del cluster c:
         Buscar en restricciones la pareja (n,elemento i)
         Si existe Y valor es -1:
            incumplidas++

      Para los cluster del conjunto distintos de c:
         Para cada elemento i del cluster j:
            Buscar en restricciones la pareja (n,elemento i)
            Si existe Y valor es 1:
               incumplidas++

      Devolver incumplidas

   \end{minted}
\end{itemize}


\section{K-medias Restringido Débil}
%MAX 2 PAGS
%Descripción en pseudocódigo de la estructura del método de búsqueda y de todas aquellas operaciones relevantes de cada algoritmo. Este contenido, específico a cada algoritmo se detallará en los correspondientes guiones de prácticas. El pseudocódigo deberá forzosamente reflejar la implementación/ el desarrollo realizados y no ser una descripción genérica extraída de las transparencias de clase o de cualquier otra fuente. La descripción de cada algoritmo no deberá ocupar más de 2 páginas.
%Para esta primera práctica se incluirá la descripción en pseudocódigo del método de exploración del entorno, el operador de generación de vecino y la generación de soluciones aleatorias empleadas en el algoritmo de BL.
\subsection{Operadores propios}
\begin{itemize}
   \item Operador \textbf{distancia\_nodo\_cluster(elemento n, cluster c)}:
   \begin{minted}
   [fontsize=\footnotesize, linenos]
   {cpp}
      double: d_euclidea, componente = 0

      Para cada caracteristica c, u del elemento n, centroide c:
         componente = abs(c - u) * abs(c - u)
         d_euclidea += componente

      Devolver d_euclidea
   \end{minted}
   \item Operador \textbf{asignar\_cluster(elemento n)}:
   \begin{minted}
   [fontsize=\footnotesize, linenos]
   {cpp}
      int: c, r_min, d_min, d = 0
      Pareja(int,int): vector[] r

      Para cada cluster i:
         Añadir la pareja (restricciones_incumplidas(n,i) , i) a r

      Ordenar r en orden ascendente

      r_min = r[0]
      d_min = Infinito
      Para cada indice j de 0...tamaño(r)
         d = distancia_nodo_cluster(n, r[j].segundo)
         Si d < d_min:
            d_min = d
            c = r[j].segundo

      infactibilidad += r_min
      Añadir n al cluster c
   \end{minted}
\end{itemize}

\subsection{Pseudocódigo del algoritmo}
\begin{minted}
[fontsize=\footnotesize, linenos]
{cpp}
   int: i = 0
   bool: cambio_c
   int: vector[] rsi
   double: matriz[][] solucion_ant

   Guardar en solucion_ant la matriz cluster

   Para cada numero de 0...i:
      Añadir numero a vector rsi
   Ordenar rsi aleatoriamente

   Do:
      cambio_c = falso
      Para cada indice j en rsi:
         asignar_cluster(j)
      Para cada cluster c:
         Si fila c de solucion_ant es distinta a la fila c de clusters:
            calcular_centroide(c)
            cambio_c = verdadero
      Guardar en solucion_ant la matriz cluster

      Si cambio_c:
         limpiar_clusters()

      i++
   While: cambio_c

   desviacion_general()
   f_objetivo = desv_gen + infactibilidad
\end{minted}


\section{Búsqueda Local}
%MAX 2 PAGS
%Descripción en pseudocódigo de la estructura del método de búsqueda y de todas aquellas operaciones relevantes de cada algoritmo. Este contenido, específico a cada algoritmo se detallará en los correspondientes guiones de prácticas. El pseudocódigo deberá forzosamente reflejar la implementación/ el desarrollo realizados y no ser una descripción genérica extraída de las transparencias de clase o de cualquier otra fuente. La descripción de cada algoritmo no deberá ocupar más de 2 páginas.
%Para esta primera práctica se incluirá la descripción en pseudocódigo del método de exploración del entorno, el operador de generación de vecino y la generación de soluciones aleatorias empleadas en el algoritmo de BL.

\section{Descripción de los algoritmos de comparación}
%Descripción en pseudocódigo de los algoritmos de comparación.

\section{Procedimiento}
%Breve explicación del procedimiento considerado para desarrollar la práctica: implementación a partir del código proporcionado en prácticas o a partir de cualquier otro, o uso de un framework de metaheurísticas concreto. Inclusión de un pequeño manual de usuario describiendo el proceso para que el profesor de prácticas pueda replicarlo.

\section{Experimentos y analisis de resultados}
\subsection{subsection name}
%Descripción de los casos del problema empleados y de los valores de los parámetros considerados en las ejecuciones de cada algoritmo (incluyendo las semillas utilizadas).

\subsection{Resultados}
% Please add the following required packages to your document preamble:
% \usepackage{graphicx}
\begin{table}[H]
\resizebox{\textwidth}{!}{%
\begin{tabular}{lcccccccccccc}
\hline
\multicolumn{13}{c}{Tabla 1: Resultados obtenidos por el algoritmo greedy en el PAR con 10\% de restricciones} \\
\multicolumn{1}{c}{} & \multicolumn{4}{c}{\textbf{Iris}} & \multicolumn{4}{c}{\textbf{Ecoli}} & \multicolumn{4}{c}{\textbf{Rand}} \\
\multicolumn{1}{c}{} & \textbf{Tasa\_C} & \textbf{Tasa\_inf} & \textbf{Agr.} & \textbf{T} & \textbf{Tasa\_C} & \textbf{Tasa\_inf} & \textbf{Agr.} & \textbf{T} & \textbf{Tasa\_C} & \textbf{Tasa\_inf} & \textbf{Agr.} & \textbf{T} \\
\textbf{Ejecución 1} & 0,60 & 0 & 0,60 & 0,01 & 1768,21 & 171 & 1.939,21 & 0,78 & 1,40 & 39 & 40,40 & 0,00 \\
\textbf{Ejecución 2} & 1,30 & 160 & 161,30 & 0,01 & 1.765,04 & 180 & 1.945,04 & 0,19 & 0,85 & 0 & 0,85 & 0,00 \\
\textbf{Ejecución 3} & 0,57 & 24 & 24,57 & 0,01 & 1.576,34 & 196 & 1.772,34 & 0,15 & 0,85 & 0 & 0,85 & 0,00 \\
\textbf{Ejecución 4} & 0,60 & 0 & 0,60 & 0,01 & 1.494,08 & 271 & 1.765,08 & 0,39 & 0,85 & 0 & 0,85 & 0,00 \\
\textbf{Ejecución 5} & 0,59 & 8 & 8,59 & 0,01 & 1.557,95 & 238 & 1.795,95 & 0,45 & 0,85 & 0 & 0,85 & 0,00 \\
\textbf{Media} & 0,73 & 38 & 39,13 & 0,01 & 1.632,32 & 211 & 1.843,52 & 0,39 & 0,96 & 8 & 8,76 & 0,00 \\ \hline
\end{tabular}%
}

% Please add the following required packages to your document preamble:
% \usepackage{graphicx}
\resizebox{\textwidth}{!}{%
\begin{tabular}{lcccccccccccc}
\multicolumn{13}{c}{Tabla 2: Resultados obtenidos por el algoritmo greedy en el PAR con 20\% de restricciones} \\
\multicolumn{1}{c}{} & \multicolumn{4}{c}{\textbf{Iris}} & \multicolumn{4}{c}{\textbf{Ecoli}} & \multicolumn{4}{c}{\textbf{Rand}} \\
\multicolumn{1}{c}{} & \textbf{Tasa\_C} & \textbf{Tasa\_inf} & \textbf{Agr.} & \textbf{T} & \textbf{Tasa\_C} & \textbf{Tasa\_inf} & \textbf{Agr.} & \textbf{T} & \textbf{Tasa\_C} & \textbf{Tasa\_inf} & \textbf{Agr.} & \textbf{T} \\
\textbf{Ejecución 1} & 0,60 & 0 & 0,60 & 0,01 & 1801,25 & 187 & 1.988,25 & 0,20 & 0,85 & 0 & 0,85 & 0,01 \\
\textbf{Ejecución 2} & 0,59 & 27 & 27,59 & 0,01 & 1.630,67 & 429 & 2.059,67 & 0,15 & 0,85 & 0 & 0,85 & 0,00 \\
\textbf{Ejecución 3} & 0,58 & 21 & 21,58 & 0,01 & 1.414,97 & 240 & 1.654,97 & 0,16 & 0,85 & 0 & 0,85 & 0,00 \\
\textbf{Ejecución 4} & 0,60 & 0 & 0,60 & 0,00 & 1.688,15 & 192 & 1.880,15 & 0,13 & 0,85 & 0 & 0,85 & 0,00 \\
\textbf{Ejecución 5} & 0,60 & 0 & 0,60 & 0,00 & 1.446,66 & 243 & 1.689,66 & 0,11 & 0,85 & 0 & 0,85 & 0,00 \\
\textbf{Media} & 0,59 & 10 & 10,19 & 0,01 & 1.596,34 & 258,20 & 1.854,54 & 0,15 & 0,85 & 0 & 0,85 & 0,00 \\ \hline
\end{tabular}%
}

% Please add the following required packages to your document preamble:
% \usepackage{graphicx}
\resizebox{\textwidth}{!}{%
\begin{tabular}{lcccccccccccc}
\multicolumn{13}{c}{Tabla 3: Resultados obtenidos por el algoritmo BL en el PAR con 10\% de restricciones} \\
 & \multicolumn{4}{c}{\textbf{Iris}} & \multicolumn{4}{c}{\textbf{Ecoli}} & \multicolumn{4}{c}{\textbf{Rand}} \\
 & \textbf{Tasa\_C} & \textbf{Tasa\_inf} & \textbf{Agr.} & \textbf{T} & \textbf{Tasa\_C} & \textbf{Tasa\_inf} & \textbf{Agr.} & \textbf{T} & \textbf{Tasa\_C} & \textbf{Tasa\_inf} & \textbf{Agr.} & \textbf{T} \\
\textbf{Ejecución 1} & 0,60 & 268 & 6,02 & 0,02 & 722,95 & 459 & 1.555,24 & 1,02 & 1,97 & 259 & 9,49 & 0,03 \\
\textbf{Ejecución 2} & 0,60 & 268 & 6,02 & 0,02 & 641,01 & 482 & 1.515,00 & 1,10 & 1,97 & 259 & 9,49 & 0,03 \\
\textbf{Ejecución 3} & 0,60 & 268 & 6,02 & 0,02 & 718,64 & 469 & 1.569,06 & 1,19 & 1,97 & 259 & 9,49 & 0,03 \\
\textbf{Ejecución 4} & 0,60 & 268 & 6,02 & 0,03 & 647,17 & 481 & 1.519,35 & 1,11 & 1,97 & 259 & 9,49 & 0,04 \\
\textbf{Ejecución 5} & 0,60 & 268 & 6,02 & 0,03 & 726,67 & 457 & 1.555,33 & 1,08 & 4,29 & 242 & 11,31 & 0,03 \\
\textbf{Media} & 0,60 & 268 & 6,02 & 0,03 & 691,29 & 470 & 1.542,80 & 1,10 & 2,43 & 256 & 9,85 & 0,03 \\ \hline
\end{tabular}%
}

% Please add the following required packages to your document preamble:
% \usepackage{graphicx}
\resizebox{\textwidth}{!}{%
\begin{tabular}{lcccccccccccc}
\multicolumn{13}{c}{Tabla 4: Resultados obtenidos por el algoritmo greedy en el PAR con 20\% de restricciones} \\
\multicolumn{1}{c}{} & \multicolumn{4}{c}{\textbf{Iris}} & \multicolumn{4}{c}{\textbf{Ecoli}} & \multicolumn{4}{c}{\textbf{Rand}} \\
\multicolumn{1}{c}{} & \textbf{Tasa\_C} & \textbf{Tasa\_inf} & \textbf{Agr.} & \textbf{T} & \textbf{Tasa\_C} & \textbf{Tasa\_inf} & \textbf{Agr.} & \textbf{T} & \textbf{Tasa\_C} & \textbf{Tasa\_inf} & \textbf{Agr.} & \textbf{T} \\
\textbf{Ejecución 1} & 0,56 & 517 & 6,36 & 0,03 & 658,114 & 1.116 & 1.669,91 & 1,14 & 1,97 & 543 & 9,85 & 0,03 \\
\textbf{Ejecución 2} & 0,56 & 517 & 6,36 & 0,03 & 657,953 & 1.124 & 1.781,95 & 1,63 & 3,77 & 500 & 11,03 & 0,05 \\
\textbf{Ejecución 3} & 2,59 & 502 & 8,23 & 0,04 & 658,114 & 1.116 & 1.774,11 & 1,04 & 1,97 & 543 & 9,85 & 0,03 \\
\textbf{Ejecución 4} & 0,56 & 517 & 6,36 & 0,03 & 643,756 & 1.067 & 1.710,76 & 1,33 & 1,97 & 543 & 9,85 & 0,03 \\
\textbf{Ejecución 5} & 0,56 & 517 & 6,36 & 0,03 & 736,222 & 1.006 & 1.742,22 & 1,74 & 4,03 & 504 & 11,34 & 0,05 \\
\textbf{Media} & 0,97 & 514 & 6,74 & 0,03 & 670,832 & 1.086 & 1.735,79 & 1,38 & 2,74 & 527 & 10,38 & 0,04 \\ \hline
\end{tabular}%
}

% Please add the following required packages to your document preamble:
% \usepackage{graphicx}
\resizebox{\textwidth}{!}{%
\begin{tabular}{lcccccccccccc}
\multicolumn{13}{c}{Tabla 5: Resultados globales en el PAR con 10\% de restricciones} \\
 & \multicolumn{4}{c}{\textbf{Iris}} & \multicolumn{4}{c}{\textbf{Ecoli}} & \multicolumn{4}{c}{\textbf{Rand}} \\
 & \textbf{Tasa\_C} & \textbf{Tasa\_inf} & \textbf{Agr.} & \textbf{T} & \textbf{Tasa\_C} & \textbf{Tasa\_inf} & \textbf{Agr.} & \textbf{T} & \textbf{Tasa\_C} & \textbf{Tasa\_inf} & \textbf{Agr.} & \textbf{T} \\
\textbf{K-medias Restringido Débil} & 0,73 & 38 & 39,13 & 0,006 & 1.632,32 & 211 & 1.843,52 & 0,389 & 0,96 & 8 & 8,76 & 0,004 \\
\textbf{Búsqueda Local} & 0,60 & 268 & 6,02 & 0,026 & 691,29 & 470 & 1.542,80 & 1,101 & 2,43 & 256 & 9,85 & 0,033 \\ \hline
\end{tabular}%
}

% Please add the following required packages to your document preamble:
% \usepackage{graphicx}
\resizebox{\textwidth}{!}{%
\begin{tabular}{lcccccccccccc}
\multicolumn{13}{c}{Tabla 6: Resultados globales en el PAR con 20\% de restricciones} \\
 & \multicolumn{4}{c}{\textbf{Iris}} & \multicolumn{4}{c}{\textbf{Ecoli}} & \multicolumn{4}{c}{\textbf{Rand}} \\
 & \textbf{Tasa\_C} & \textbf{Tasa\_inf} & \textbf{Agr.} & \textbf{T} & \textbf{Tasa\_C} & \textbf{Tasa\_inf} & \textbf{Agr.} & \textbf{T} & \textbf{Tasa\_C} & \textbf{Tasa\_inf} & \textbf{Agr.} & \textbf{T} \\
\textbf{K-medias Restringido Débil} & 0,59 & 10 & 10,19 & 0,006 & 1596,34 & 258 & 1854,54 & 0,152 & 0,85 & 0 & 0,85 & 0,005 \\
\textbf{Búsqueda Local} & 0,97 & 514 & 6,74 & 0,033 & 670,83 & 1086 & 1735,79 & 1,377 & 2,74 & 526 & 10,38 & 0,039 \\ \hline
\end{tabular}%
}
\end{table}

\subsection{subsection name}
%Análisis de resultados. El análisis deberá estar orientado a justificar (según el comportamiento de cada algoritmo) los resultados obtenidos en lugar de realizar una mera “lectura” de las tablas. Se valorará la inclusión de otros elementos de comparación tales como gráficas de convergencia, boxplots, análisis comparativo de las soluciones obtenidas, representación gráfica de las soluciones, etc.

\section{Bibliografia}
