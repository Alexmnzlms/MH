\chapter{Técnicas de Búsqueda Local y Algoritmos Greedy para el Problema del Agrupamiento con Restricciones}

\section{Descripción del problema}
Breve descripción/formulación del problema (máximo 1 página). Podrá incluirse
el mismo contenido repetido en todas las prácticas presentadas por el estudiante.

\section{Descripción de los algoritmos empleados}
Breve descripción de la aplicación de los algoritmos empleados al problema
(máximo 4 páginas): Todas las consideraciones comunes a los distintos
algoritmos se describirán en este apartado, que será previo a la descripción de los
algoritmos específicos. Incluirá por ejemplo la descripción del esquema de
representación de soluciones y la descripción en pseudocódigo (no código) de la
función objetivo y los operadores comunes.

\section{K-medias Restringido Débil}
Descripción en pseudocódigo de la estructura del método de búsqueda y de
todas aquellas operaciones relevantes de cada algoritmo. Este contenido,
específico a cada algoritmo se detallará en los correspondientes guiones de
prácticas. El pseudocódigo deberá forzosamente reflejar la implementación/ el
desarrollo realizados y no ser una descripción genérica extraída de las
transparencias de clase o de cualquier otra fuente. La descripción de cada
algoritmo no deberá ocupar más de 2 páginas.
Para esta primera práctica se incluirá la descripción en pseudocódigo del método
de exploración del entorno, el operador de generación de vecino y la generación
de soluciones aleatorias empleadas en el algoritmo de BL.

\section{Búsqueda Local}
Descripción en pseudocódigo de la estructura del método de búsqueda y de
todas aquellas operaciones relevantes de cada algoritmo. Este contenido,
específico a cada algoritmo se detallará en los correspondientes guiones de
prácticas. El pseudocódigo deberá forzosamente reflejar la implementación/ el
desarrollo realizados y no ser una descripción genérica extraída de las
transparencias de clase o de cualquier otra fuente. La descripción de cada
algoritmo no deberá ocupar más de 2 páginas.
Para esta primera práctica se incluirá la descripción en pseudocódigo del método
de exploración del entorno, el operador de generación de vecino y la generación
de soluciones aleatorias empleadas en el algoritmo de BL.

\section{Descripción de los algoritmos de comparación}
Descripción en pseudocódigo de los algoritmos de comparación.

\section{Procedimiento}
Breve explicación del procedimiento considerado para desarrollar la práctica:
implementación a partir del código proporcionado en prácticas o a partir de
cualquier otro, o uso de un framework de metaheurísticas concreto. Inclusión de
un pequeño manual de usuario describiendo el proceso para que el profesor
de prácticas pueda replicarlo.

\section{Experimentos y analisis de resultados}
\subsection{subsection name}
Descripción de los casos del problema empleados y de los valores de los
parámetros considerados en las ejecuciones de cada algoritmo
(incluyendo las semillas utilizadas).
\subsection{subsection name}
Resultados obtenidos según el formato especificado.

\subsection{Tablas}
% Please add the following required packages to your document preamble:
% \usepackage{graphicx}
\begin{table}[]
\resizebox{\textwidth}{!}{%
\begin{tabular}{lcccccccccccc}
\multicolumn{13}{c}{Tabla 1: Resultados obtenidos por el algoritmo greedy en el PAR con 10\% de restricciones} \\
\multicolumn{1}{c}{} & \multicolumn{4}{c}{\textbf{Iris}} & \multicolumn{4}{c}{\textbf{Ecoli}} & \multicolumn{4}{c}{\textbf{Rand}} \\
\multicolumn{1}{c}{} & \textbf{Tasa\_C} & \textbf{Tasa\_inf} & \textbf{Agr.} & \textbf{T} & \textbf{Tasa\_C} & \textbf{Tasa\_inf} & \textbf{Agr.} & \textbf{T} & \textbf{Tasa\_C} & \textbf{Tasa\_inf} & \textbf{Agr.} & \textbf{T} \\
\textbf{Ejecución 1} & 0,60 & 0 & 0,60 & 0,00503 & 1768,21 & 171 & 1.939,21 & 0,78 & 1,40 & 39 & 40,40 & 0,00 \\
\textbf{Ejecución 2} & 1,30 & 160 & 161,30 & 0,00517 & 1.765,04 & 180 & 1.945,04 & 0,19 & 0,85 & 0 & 0,85 & 0,00 \\
\textbf{Ejecución 3} & 0,57 & 24 & 24,57 & 0,00521 & 1.576,34 & 196 & 1.772,34 & 0,15 & 0,85 & 0 & 0,85 & 0,00 \\
\textbf{Ejecución 4} & 0,60 & 0 & 0,60 & 0,00524 & 1.494,08 & 271 & 1.765,08 & 0,39 & 0,85 & 0 & 0,85 & 0,00 \\
\textbf{Ejecución 5} & 0,59 & 8 & 8,59 & 0,00740 & 1.557,95 & 238 & 1.795,95 & 0,45 & 0,85 & 0 & 0,85 & 0,00 \\
\textbf{Media} & 0,73 & 38 & 39,13 & 0,00561 & 1.632,32 & 211 & 1.843,52 & 0,39 & 0,96 & 8 & 8,76 & 0,00
\end{tabular}%
}
\end{table}



% Please add the following required packages to your document preamble:
% \usepackage{graphicx}
\begin{table}[]
\resizebox{\textwidth}{!}{%
\begin{tabular}{lcccccccccccc}
\multicolumn{13}{c}{Tabla 2: Resultados obtenidos por el algoritmo greedy en el PAR con 20\% de restricciones} \\
\multicolumn{1}{c}{} & \multicolumn{4}{c}{\textbf{Iris}} & \multicolumn{4}{c}{\textbf{Ecoli}} & \multicolumn{4}{c}{\textbf{Rand}} \\
\multicolumn{1}{c}{} & \textbf{Tasa\_C} & \textbf{Tasa\_inf} & \textbf{Agr.} & \textbf{T} & \textbf{Tasa\_C} & \textbf{Tasa\_inf} & \textbf{Agr.} & \textbf{T} & \textbf{Tasa\_C} & \textbf{Tasa\_inf} & \textbf{Agr.} & \textbf{T} \\
\textbf{Ejecución 1} & 0,60 & 0 & 0,60 & 0,00603 & 1801,25 & 187 & 1.988,25 & 0,20 & 0,85 & 0 & 0,85 & 0,01 \\
\textbf{Ejecución 2} & 0,59 & 27 & 27,59 & 0,00552 & 1.630,67 & 429 & 2.059,67 & 0,15 & 0,85 & 0 & 0,85 & 0,00 \\
\textbf{Ejecución 3} & 0,58 & 21 & 21,58 & 0,00537 & 1.414,97 & 240 & 1.654,97 & 0,16 & 0,85 & 0 & 0,85 & 0,00 \\
\textbf{Ejecución 4} & 0,60 & 0 & 0,60 & 0,00608 & 1.688,15 & 192 & 1.880,15 & 0,13 & 0,85 & 0 & 0,85 & 0,00 \\
\textbf{Ejecución 5} & 0,60 & 0 & 0,60 & 0,00488 & 1.446,66 & 243 & 1.689,66 & 0,11 & 0,85 & 0 & 0,85 & 0,00 \\
\textbf{Media} & 0,59 & 10 & 10,19 & 0,01 & 1.596,34 & 258,20 & 1.854,54 & 0,15 & 0,85 & 0 & 0,85 & 0,00
\end{tabular}%
}
\end{table}



% Please add the following required packages to your document preamble:
% \usepackage{graphicx}
\begin{table}[]
\resizebox{\textwidth}{!}{%
\begin{tabular}{lcccccccccccc}
\multicolumn{13}{c}{Tabla 3: Resultados obtenidos por el algoritmo BL en el PAR con 10\% de restricciones} \\
 & \multicolumn{4}{c}{\textbf{Iris}} & \multicolumn{4}{c}{\textbf{Ecoli}} & \multicolumn{4}{c}{\textbf{Rand}} \\
 & \textbf{Tasa\_C} & \textbf{Tasa\_inf} & \textbf{Agr.} & \textbf{T} & \textbf{Tasa\_C} & \textbf{Tasa\_inf} & \textbf{Agr.} & \textbf{T} & \textbf{Tasa\_C} & \textbf{Tasa\_inf} & \textbf{Agr.} & \textbf{T} \\
\textbf{Ejecución 1} & 0,60 & 268 & 6,02 & 0,02793 & 722,95 & 459 & 1.555,24 & 1,02 & 1,97 & 259 & 9,49 & 0,03 \\
\textbf{Ejecución 2} & 0,60 & 268 & 6,02 & 0,02688 & 641,01 & 482 & 1.515,00 & 1,10 & 1,97 & 259 & 9,49 & 0,03 \\
\textbf{Ejecución 3} & 0,60 & 268 & 6,02 & 0,02451 & 718,64 & 469 & 1.569,06 & 1,19 & 1,97 & 259 & 9,49 & 0,03 \\
\textbf{Ejecución 4} & 0,60 & 268 & 6,02 & 0,02647 & 647,17 & 481 & 1.519,35 & 1,11 & 1,97 & 259 & 9,49 & 0,04 \\
\textbf{Ejecución 5} & 0,60 & 268 & 6,02 & 0,02543 & 726,67 & 457 & 1.555,33 & 1,08 & 4,29 & 242 & 11,31 & 0,03 \\
\textbf{Media} & 0,60 & 268 & 6,02 & 0,03 & 691,29 & 470 & 1.542,80 & 1,10 & 2,43 & 256 & 9,85 & 0,03
\end{tabular}%
}
\end{table}



% Please add the following required packages to your document preamble:
% \usepackage{graphicx}
\begin{table}[]
\resizebox{\textwidth}{!}{%
\begin{tabular}{lcccccccccccc}
\multicolumn{13}{c}{Tabla 4: Resultados obtenidos por el algoritmo greedy en el PAR con 20\% de restricciones} \\
\multicolumn{1}{c}{} & \multicolumn{4}{c}{\textbf{Iris}} & \multicolumn{4}{c}{\textbf{Ecoli}} & \multicolumn{4}{c}{\textbf{Rand}} \\
\multicolumn{1}{c}{} & \textbf{Tasa\_C} & \textbf{Tasa\_inf} & \textbf{Agr.} & \textbf{T} & \textbf{Tasa\_C} & \textbf{Tasa\_inf} & \textbf{Agr.} & \textbf{T} & \textbf{Tasa\_C} & \textbf{Tasa\_inf} & \textbf{Agr.} & \textbf{T} \\
\textbf{Ejecución 1} & 0,56 & 517 & 6,36 & 0,03022 & 658,114 & 1.116 & 1.669,91 & 1,14 & 1,97 & 543 & 9,85 & 0,03 \\
\textbf{Ejecución 2} & 0,56 & 517 & 6,36 & 0,03385 & 657,953 & 1.124 & 1.781,95 & 1,63 & 3,77 & 500 & 11,03 & 0,05 \\
\textbf{Ejecución 3} & 2,59 & 502 & 8,23 & 0,03836 & 658,114 & 1.116 & 1.774,11 & 1,04 & 1,97 & 543 & 9,85 & 0,03 \\
\textbf{Ejecución 4} & 0,56 & 517 & 6,36 & 0,03218 & 643,756 & 1.067 & 1.710,76 & 1,33 & 1,97 & 543 & 9,85 & 0,03 \\
\textbf{Ejecución 5} & 0,56 & 517 & 6,36 & 0,03254 & 736,222 & 1.006 & 1.742,22 & 1,74 & 4,03 & 504 & 11,34 & 0,05 \\
\textbf{Media} & 0,97 & 514 & 6,74 & 0,03343 & 670,832 & 1.086 & 1.735,79 & 1,38 & 2,74 & 527 & 10,38 & 0,04
\end{tabular}%
}
\end{table}

% Please add the following required packages to your document preamble:
% \usepackage{graphicx}
\begin{table}[]
\centering
\resizebox{\textwidth}{!}{%
\begin{tabular}{lcccccccccccc}
\multicolumn{13}{c}{Tabla 5: Resultados globales en el PAR con 10\% de restricciones} \\
 & \multicolumn{4}{c}{\textbf{Iris}} & \multicolumn{4}{c}{\textbf{Ecoli}} & \multicolumn{4}{c}{\textbf{Rand}} \\
 & \textbf{Tasa\_C} & \textbf{Tasa\_inf} & \textbf{Agr.} & \textbf{T} & \textbf{Tasa\_C} & \textbf{Tasa\_inf} & \textbf{Agr.} & \textbf{T} & \textbf{Tasa\_C} & \textbf{Tasa\_inf} & \textbf{Agr.} & \textbf{T} \\
\textbf{K-medias Restringido Débil} & 0,73 & 38 & 39,13 & 0,006 & 1.632,32 & 211 & 1.843,52 & 0,389 & 0,96 & 8 & 8,76 & 0,004 \\
\textbf{Búsqueda Local} & 0,60 & 268 & 6,02 & 0,026 & 691,29 & 470 & 1.542,80 & 1,101 & 2,43 & 256 & 9,85 & 0,033
\end{tabular}%
}
\end{table}

% Please add the following required packages to your document preamble:
% \usepackage{graphicx}
\begin{table}[]
\centering
\resizebox{\textwidth}{!}{%
\begin{tabular}{lcccccccccccc}
\multicolumn{13}{c}{Tabla 6: Resultados globales en el PAR con 20\% de restricciones} \\
 & \multicolumn{4}{c}{\textbf{Iris}} & \multicolumn{4}{c}{\textbf{Ecoli}} & \multicolumn{4}{c}{\textbf{Rand}} \\
 & \textbf{Tasa\_C} & \textbf{Tasa\_inf} & \textbf{Agr.} & \textbf{T} & \textbf{Tasa\_C} & \textbf{Tasa\_inf} & \textbf{Agr.} & \textbf{T} & \textbf{Tasa\_C} & \textbf{Tasa\_inf} & \textbf{Agr.} & \textbf{T} \\
\textbf{K-medias Restringido Débil} & 0,591376 & 9,6 & 10,191376 & 0,006 & 1596,34 & 258,2 & 1854,54 & 0,152 & 0,85217 & 0 & 0,85217 & 0,005 \\
\textbf{Búsqueda Local} & 0,9650284 & 514 & 6,737454 & 0,033 & 670,8318 & 1085,8 & 1735,79117 & 1,377 & 2,741634 & 526,6 & 10,38302128 & 0,039
\end{tabular}%
}
\end{table}

\subsection{subsection name}
Análisis de resultados. El análisis deberá estar orientado a justificar
(según el comportamiento de cada algoritmo) los resultados obtenidos en
lugar de realizar una mera “lectura” de las tablas. Se valorará la inclusión
de otros elementos de comparación tales como gráficas de convergencia,
boxplots, análisis comparativo de las soluciones obtenidas, representación
gráfica de las soluciones, etc.

\section{Bibliografia}
