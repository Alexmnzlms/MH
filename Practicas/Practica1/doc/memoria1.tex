\newpage
\chapter{Procedimiento}
\section{Estructura de datos}
%Breve explicación del procedimiento considerado para desarrollar la práctica: implementación a partir del código proporcionado en prácticas o a partir de cualquier otro, o uso de un framework de metaheurísticas concreto. Inclusión de un pequeño manual de usuario describiendo el proceso para que el profesor de prácticas pueda replicarlo.
La implementación de la práctica se ha llevado a cabo en c++.\\
Para la estructura de datos he optado por una sola clase, llamada CCP --- Constrained Clustering Problem --- en la que estan todos los datos necesarios para realizar el problema:
\begin{minted}
[fontsize=\footnotesize, linenos]
{cpp}
int n_cluster;
std::vector<std::vector<double>> posiciones;
std::vector<std::vector<double>> centroides;
std::map<std::pair<int,int>,int> restricciones;
std::set<std::pair<int,int>> vecindario;
std::vector<std::vector<int>> clusters;
std::vector<double> d_intracluster;
std::vector<int> solucion;
double desv_gen;
double infactibilidad;
double lambda;
double f_objetivo;
\end{minted}

He utilizado las clases map, set y vector de la STL, ya que son muy eficientes y tienen implementados funciones muy utiles.\\
Tambien los operadores de los algoritmos se han implementado como metodos de la clase CCP:
\begin{minted}
[fontsize=\footnotesize, linenos]
{cpp}
void cargar_posiciones(const std::string archivo);
void cargar_restricciones(const std::string archivo);
void calcular_centroide(const int i);
void distancia_intracluster(const int i);
void desviacion_general();
void leer_solucion();
void leer_vecino();
void generar_solucion();
void asignar_cluster(const int n);
double distancia_nodo_cluster(const int n, const int c);
double restricciones_incumplidas(const int n, const int c);
void limpiar_clusters();
void generar_vecino();
int buscar_cluster(const int n);
void solucion_inicial();
void calcular_lambda();
double distancia_nodo_nodo(const int n, const int m);
void generar_vecindario();
bool quedan_vecinos();
\end{minted}

\section{Guía de Uso}

El problema el muy sencillo de compilar. Con la orden \textbf{make} el programa se compila y se ejecuta autimaticamente. Los datos se cargan automaticamente desde los ficheros siempre que se encuentren en una carpeta data/\\
La estructura de ficheros es la siguiente:
\begin{itemize}
   \item cc.h: Cabecera de la clase CCP.
   \item cc.p: implementación de los metodos de la clase CCP.
   \item main.cpp: Implentación de la ejecución de los algoritmos greedy y BL.
\end{itemize}



\newpage
\chapter{Experimentos y analisis de resultados}
\section{section name}
%Descripción de los casos del problema empleados y de los valores de los parámetros considerados en las ejecuciones de cada algoritmo (incluyendo las semillas utilizadas).

\section{Resultados}
% Please add the following required packages to your document preamble:
% \usepackage{graphicx}
\begin{table}[H]
\resizebox{\textwidth}{!}{%
\begin{tabular}{lcccccccccccc}
\hline
\multicolumn{13}{c}{Tabla 1: Resultados obtenidos por el algoritmo greedy en el PAR con 10\% de restricciones} \\
\multicolumn{1}{c}{} & \multicolumn{4}{c}{\textbf{Iris}} & \multicolumn{4}{c}{\textbf{Ecoli}} & \multicolumn{4}{c}{\textbf{Rand}} \\
\multicolumn{1}{c}{} & \textbf{Tasa\_C} & \textbf{Tasa\_inf} & \textbf{Agr.} & \textbf{T} & \textbf{Tasa\_C} & \textbf{Tasa\_inf} & \textbf{Agr.} & \textbf{T} & \textbf{Tasa\_C} & \textbf{Tasa\_inf} & \textbf{Agr.} & \textbf{T} \\
\textbf{Ejecución 1} & 0,60 & 0 & 0,60 & 0,01 & 1768,21 & 171 & 1.939,21 & 0,78 & 1,40 & 39 & 40,40 & 0,00 \\
\textbf{Ejecución 2} & 1,30 & 160 & 161,30 & 0,01 & 1.765,04 & 180 & 1.945,04 & 0,19 & 0,85 & 0 & 0,85 & 0,00 \\
\textbf{Ejecución 3} & 0,57 & 24 & 24,57 & 0,01 & 1.576,34 & 196 & 1.772,34 & 0,15 & 0,85 & 0 & 0,85 & 0,00 \\
\textbf{Ejecución 4} & 0,60 & 0 & 0,60 & 0,01 & 1.494,08 & 271 & 1.765,08 & 0,39 & 0,85 & 0 & 0,85 & 0,00 \\
\textbf{Ejecución 5} & 0,59 & 8 & 8,59 & 0,01 & 1.557,95 & 238 & 1.795,95 & 0,45 & 0,85 & 0 & 0,85 & 0,00 \\
\textbf{Media} & 0,73 & 38 & 39,13 & 0,01 & 1.632,32 & 211 & 1.843,52 & 0,39 & 0,96 & 8 & 8,76 & 0,00 \\ \hline
\end{tabular}%
}

% Please add the following required packages to your document preamble:
% \usepackage{graphicx}
\resizebox{\textwidth}{!}{%
\begin{tabular}{lcccccccccccc}
\multicolumn{13}{c}{Tabla 2: Resultados obtenidos por el algoritmo greedy en el PAR con 20\% de restricciones} \\
\multicolumn{1}{c}{} & \multicolumn{4}{c}{\textbf{Iris}} & \multicolumn{4}{c}{\textbf{Ecoli}} & \multicolumn{4}{c}{\textbf{Rand}} \\
\multicolumn{1}{c}{} & \textbf{Tasa\_C} & \textbf{Tasa\_inf} & \textbf{Agr.} & \textbf{T} & \textbf{Tasa\_C} & \textbf{Tasa\_inf} & \textbf{Agr.} & \textbf{T} & \textbf{Tasa\_C} & \textbf{Tasa\_inf} & \textbf{Agr.} & \textbf{T} \\
\textbf{Ejecución 1} & 0,60 & 0 & 0,60 & 0,01 & 1801,25 & 187 & 1.988,25 & 0,20 & 0,85 & 0 & 0,85 & 0,01 \\
\textbf{Ejecución 2} & 0,59 & 27 & 27,59 & 0,01 & 1.630,67 & 429 & 2.059,67 & 0,15 & 0,85 & 0 & 0,85 & 0,00 \\
\textbf{Ejecución 3} & 0,58 & 21 & 21,58 & 0,01 & 1.414,97 & 240 & 1.654,97 & 0,16 & 0,85 & 0 & 0,85 & 0,00 \\
\textbf{Ejecución 4} & 0,60 & 0 & 0,60 & 0,00 & 1.688,15 & 192 & 1.880,15 & 0,13 & 0,85 & 0 & 0,85 & 0,00 \\
\textbf{Ejecución 5} & 0,60 & 0 & 0,60 & 0,00 & 1.446,66 & 243 & 1.689,66 & 0,11 & 0,85 & 0 & 0,85 & 0,00 \\
\textbf{Media} & 0,59 & 10 & 10,19 & 0,01 & 1.596,34 & 258,20 & 1.854,54 & 0,15 & 0,85 & 0 & 0,85 & 0,00 \\ \hline
\end{tabular}%
}

% Please add the following required packages to your document preamble:
% \usepackage{graphicx}
\resizebox{\textwidth}{!}{%
\begin{tabular}{lcccccccccccc}
\multicolumn{13}{c}{Tabla 3: Resultados obtenidos por el algoritmo BL en el PAR con 10\% de restricciones} \\
 & \multicolumn{4}{c}{\textbf{Iris}} & \multicolumn{4}{c}{\textbf{Ecoli}} & \multicolumn{4}{c}{\textbf{Rand}} \\
 & \textbf{Tasa\_C} & \textbf{Tasa\_inf} & \textbf{Agr.} & \textbf{T} & \textbf{Tasa\_C} & \textbf{Tasa\_inf} & \textbf{Agr.} & \textbf{T} & \textbf{Tasa\_C} & \textbf{Tasa\_inf} & \textbf{Agr.} & \textbf{T} \\
\textbf{Ejecución 1} & 0,60 & 268 & 6,63 & 0,02 & 722,95 & 459 & 1.652,70 & 1,02 & 1,97 & 259 & 9,49 & 0,03 \\
\textbf{Ejecución 2} & 0,60 & 268 & 6,63 & 0,02 & 641,01 & 482 & 1.617,34 & 1,10 & 1,97 & 259 & 9,49 & 0,03 \\
\textbf{Ejecución 3} & 0,60 & 268 & 6,63 & 0,02 & 718,64 & 469 & 1.668,65 & 1,19 & 1,97 & 259 & 9,49 & 0,03 \\
\textbf{Ejecución 4} & 0,60 & 268 & 6,63 & 0,03 & 647,17 & 481 & 1.621,48 & 1,11 & 1,97 & 259 & 9,49 & 0,04 \\
\textbf{Ejecución 5} & 0,60 & 268 & 6,63 & 0,03 & 726,67 & 457 & 1.652,37 & 1,08 & 4,29 & 242 & 11,31 & 0,03 \\
\textbf{Media} & 0,60 & 268 & 6,63 & 0,03 & 691,29 & 470 & 1.642,51 & 1,10 & 2,43 & 256 & 9,85 & 0,03 \\ \hline
\end{tabular}%
}

% Please add the following required packages to your document preamble:
% \usepackage{graphicx}
\resizebox{\textwidth}{!}{%
\begin{tabular}{lcccccccccccc}
\multicolumn{13}{c}{Tabla 4: Resultados obtenidos por el algoritmo greedy en el PAR con 20\% de restricciones} \\
\multicolumn{1}{c}{} & \multicolumn{4}{c}{\textbf{Iris}} & \multicolumn{4}{c}{\textbf{Ecoli}} & \multicolumn{4}{c}{\textbf{Rand}} \\
\multicolumn{1}{c}{} & \textbf{Tasa\_C} & \textbf{Tasa\_inf} & \textbf{Agr.} & \textbf{T} & \textbf{Tasa\_C} & \textbf{Tasa\_inf} & \textbf{Agr.} & \textbf{T} & \textbf{Tasa\_C} & \textbf{Tasa\_inf} & \textbf{Agr.} & \textbf{T} \\
\textbf{Ejecución 1} & 0,56 & 517 & 6,36 & 0,03 & 658,114 & 1.116 & 1.746,60 & 1,14 & 1,97 & 543 & 9,85 & 0,03 \\
\textbf{Ejecución 2} & 0,56 & 517 & 6,36 & 0,03 & 657,953 & 1.124 & 1.745,24 & 1,63 & 3,77 & 500 & 11,03 & 0,05 \\
\textbf{Ejecución 3} & 2,59 & 502 & 8,23 & 0,04 & 658,114 & 1.116 & 1.746,60 & 1,04 & 1,97 & 543 & 9,85 & 0,03 \\
\textbf{Ejecución 4} & 0,56 & 517 & 6,36 & 0,03 & 643,756 & 1.067 & 1.684,45 & 1,33 & 1,97 & 543 & 9,85 & 0,03 \\
\textbf{Ejecución 5} & 0,56 & 517 & 6,36 & 0,03 & 736,222 & 1.006 & 1.717,42 & 1,74 & 4,03 & 504 & 11,34 & 0,05 \\
\textbf{Media} & 0,97 & 514 & 6,74 & 0,03 & 670,832 & 1.086 & 1.729,86 & 1,38 & 2,74 & 527 & 10,38 & 0,04 \\ \hline
\end{tabular}%
}

% Please add the following required packages to your document preamble:
% \usepackage{graphicx}
\resizebox{\textwidth}{!}{%
\begin{tabular}{lcccccccccccc}
\multicolumn{13}{c}{Tabla 5: Resultados globales en el PAR con 10\% de restricciones} \\
 & \multicolumn{4}{c}{\textbf{Iris}} & \multicolumn{4}{c}{\textbf{Ecoli}} & \multicolumn{4}{c}{\textbf{Rand}} \\
 & \textbf{Tasa\_C} & \textbf{Tasa\_inf} & \textbf{Agr.} & \textbf{T} & \textbf{Tasa\_C} & \textbf{Tasa\_inf} & \textbf{Agr.} & \textbf{T} & \textbf{Tasa\_C} & \textbf{Tasa\_inf} & \textbf{Agr.} & \textbf{T} \\
\textbf{K-medias Restringido Débil} & 0,73 & 38 & 39,13 & 0,006 & 1.632,32 & 211 & 1.843,52 & 0,389 & 0,96 & 8 & 8,76 & 0,004 \\
\textbf{Búsqueda Local} & 0,60 & 268 & 6,63 & 0,026 & 691,29 & 470 & 1.642,51 & 1,101 & 2,43 & 256 & 9,85 & 0,033 \\ \hline
\end{tabular}%
}

% Please add the following required packages to your document preamble:
% \usepackage{graphicx}
\resizebox{\textwidth}{!}{%
\begin{tabular}{lcccccccccccc}
\multicolumn{13}{c}{Tabla 6: Resultados globales en el PAR con 20\% de restricciones} \\
 & \multicolumn{4}{c}{\textbf{Iris}} & \multicolumn{4}{c}{\textbf{Ecoli}} & \multicolumn{4}{c}{\textbf{Rand}} \\
 & \textbf{Tasa\_C} & \textbf{Tasa\_inf} & \textbf{Agr.} & \textbf{T} & \textbf{Tasa\_C} & \textbf{Tasa\_inf} & \textbf{Agr.} & \textbf{T} & \textbf{Tasa\_C} & \textbf{Tasa\_inf} & \textbf{Agr.} & \textbf{T} \\
\textbf{K-medias Restringido Débil} & 0,59 & 10 & 10,19 & 0,006 & 1.596,34 & 258 & 1854,54 & 0,152 & 0,85 & 0 & 0,85 & 0,005 \\
\textbf{Búsqueda Local} & 0,97 & 514 & 6,74 & 0,033 & 670,83 & 1086 & 1.729,86 & 1,377 & 2,74 & 526 & 10,38 & 0,039 \\ \hline
\end{tabular}%
}
\end{table}

\section{section name}
%Análisis de resultados. El análisis deberá estar orientado a justificar (según el comportamiento de cada algoritmo) los resultados obtenidos en lugar de realizar una mera “lectura” de las tablas. Se valorará la inclusión de otros elementos de comparación tales como gráficas de convergencia, boxplots, análisis comparativo de las soluciones obtenidas, representación gráfica de las soluciones, etc.
